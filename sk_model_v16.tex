\documentclass[twocolumn,superscriptaddress,prb,10pt]{revtex4-1}
%\usepackage{verbatim}
\usepackage{amsmath,amssymb}
\usepackage{graphicx}
\usepackage{color}
\usepackage[colorlinks,bookmarks=false,citecolor=blue,linkcolor=red,urlcolor=blue]{hyperref}
\usepackage{times}


%\usepackage[dvips]{graphics}



%%%%%%%%%%%%%   useful shortcuts %%%%%%%%%%%%%%%%%%%%%%%%%%%%%%%%%

\def \h{\hbar}   %  \h won't be used for any greek letter
\def \refe{\eqref}
\def \trm{\textrm}
\def \f{\frac}
\def \hf{\tfrac{1}{2}}    \def \HF{\dfrac{1}{2}}
\def \u{\uparrow}
\def \d{\downarrow}

\def \ord{\mathcal{O}}
\newcommand{\ra}{\rightarrow}   \newcommand{\lra}{\longrightarrow}  

\def\lba{\left(}    \def\rba{\right)}
\def\lbc{\left[}    \def\rbc{\right]}
\def\lbb{\left\{}    \def\rbb{\right\}}

\def\tr{\textrm{Tr}}
\def\refe{\eqref}

\newcommand{\bra}[1]{\langle\left.{#1}\right|}
\newcommand{\ket}[1]{\left|{#1}\right.\rangle}
\newcommand{\xpct}[1]{\langle{#1}\rangle}    % expectatn value

%\DeclareMathOperator{\tr}{tr}

%%%%%%%%%%%%%%%%%%%%%%%%%%%%%%%%%%%%%%%%%%%%%%%%%%%%%%%

\newcommand{\vp}{{\bf p}}  % usual vector quantities
\newcommand{\vq}{{\bf q}}  % double bracketing not required with \vec
\newcommand{\vk}{{\bf k}}  % but required with \bf

\renewcommand{\vr}{{\bf r}} 
\newcommand{\vx}{{\bf x}}

\newcommand{\hc}{\hat{c}}  \newcommand{\hcd}{\hat{c}^\dag} 
\newcommand{\hd}{\hat{d}}  \newcommand{\hdd}{\hat{d}^\dag} 






\begin{document}

\title{The classical mutual information in mean-field spin glass models} 

\author{Vincenzo Alba}
\affiliation{International School for Advanced Studies (SISSA),
Via Bonomea 265, 34136, Trieste, Italy, 
INFN, Sezione di Trieste}
\author{Stephen Inglis}
\affiliation{Department of Physics and Arnold Sommerfeld
Center for Theoretical Physics, Ludwig-Maximilians-Universit\"at
M\"unchen, D-80333 M\"unchen, Germany}
\author{Lode Pollet}
\affiliation{Department of Physics and Arnold Sommerfeld
    Center for Theoretical Physics, Ludwig-Maximilians-Universit\"at
M\"unchen, D-80333 M\"unchen, Germany}

\date{\today}




\begin{abstract} 

We investigate the \emph{classical} R\'enyi entropy $S_n$ and the associated mutual 
information ${\mathcal I}_n$ in the Sherrington-Kirkpatrick (S-K) model, which is the 
paradigm model of mean-field spin glasses. 
Using classical Monte Carlo simulations and analytical tools we first investigate the 
S-K model on the $n$-sheets booklet.
This is obtained by gluing together $n$ independent copies of the model, and it is the 
main ingredient to construct the R\'enyi entanglement-related quantities. We find that at 
low temperature the S-K model is in a glassy phase, whereas at high temperature it exhibits 
paramagnetic behavior. Interestingly, the temperature of the paramagnetic-glassy 
transition depends non-trivially on the geometry of the booklet. At high-temperatures 
we provide the exact solution of the model exploiting the replica symmetry. This is the 
permutation symmetry among the fictitious replicas that are used to 
perform disorder averages (via the replica trick). In the glassy phase the replica symmetry 
has to be broken. Using a generalization of the celebrated Parisi solution, we provide 
analytical results for $S_n$ and ${\mathcal I}_n$, and for standard thermodynamic quantities. 
Both $S_n$ and ${\mathcal I}_n$ exhibit a volume law in the whole phase diagram. We fully 
characterize the behavior of the corresponding densities $S_n/N$ and ${\mathcal I}_n/N$, 
in the thermodynamic limit. Remarkably, at the critical point the mutual information does 
not exhibit any crossing for different system sizes, in contrast with local spin models. 

\end{abstract}

% \pacs{73.43.Cd, 71.10.Pm  {\tt check!}}

\maketitle


%########################################################################
\section{Introduction}


Besides being ubiquitous in nature, disorder leads to several intriguing 
physical phenomena. 
Arguably, \emph{spin glasses} represent one of the most prototypical examples
of exotic behavior induced by disorder.
While at any finite temperature disorder prevents the 
usual magnetic ordering, at low-enough temperatures these systems display a 
new type of ``order''. In the past decades an intense theoretical 
effort has been devoted to characterizing this spin glass order, the nature of 
the paramagnetic-glassy transition, and that of the associated order 
parameter~\cite{binder-1986,parisi-book,young-1998,nishimori-book,castellani-2005}. 

All these issues can be thoroughly addressed in the Sherrington-Kirkpatrick 
(S-K) model~\cite{sherrington-1978,sherrington-1978-prl}, which is exactly 
solvable. The S-K model is a \emph{classical} Ising model on the fully-connected 
graph of $N$ sites, with quenched random interactions. Its hamiltonian reads  
%
\begin{equation}
{\mathcal H}=-\sum\limits_{1\le i<j\le N}J_{ij}S_i S_j-
h\sum\limits_{1\le i\le N}S_i.
\label{SK-intro}
\end{equation}
%
Here $S_i=\pm 1$ are classical Ising spins, $h$ is an external magnetic field, and 
$J_{ij}$ are uncorrelated (from site to site) random variables. The S-K model hosts 
a low-temperature glassy phase, which is separated from a high-temperature paramagnetic 
one by a second order phase transition. Despite its mean-field nature, the solution 
of the S-K model has been a mathematical challenge. Although it was proposed as an 
ansatz by Parisi~\cite{parisi-1980} more than thirty years ago, its rigorous proof 
was obtained only recently~\cite{talagrand-2006}. Moreover, the solution exhibits several 
intricate features, such as lack of self-averaging~\cite{pastur-1991}, ultrametricity~\cite{
mezard-1984,rammal-1986}, and replica symmetry breaking~\cite{parisi-book,castellani-2005}. 
The last refers to the breaking of the permutation symmetry among the fictitious replicas 
of the model, which are introduced to perform disorder averages (via the so-called 
\emph{replica trick}~\cite{cardy-book}). Finally, although the applicability of the S-K 
model to describe realistic spin glasses is still highly debated~\cite{yucesoy-2012,
billoire-2012,yucesoy-2013}, there are recent proposals on how to realize it  
in cold-atomic gases~\cite{morrison-2008,rotondo-2015}, or in laser systems~\cite{
ghofraniha-2015}. 

In the last decade entanglement-related quantities emerged as valuable tools to 
understand the physics of complex systems~\cite{amico-2008,eisert-2009,calabrese-2009,
cc-rev}, both classical and quantum. For instance, at a conformally invariant critical 
point entanglement measures contain universal information about the underlying 
conformal field theory (CFT), such as the central charge~\cite{holzhey-1994,vidal-2003,
calabrese-2004,calabrese-2012}. For classical spin models a lot of attention has been 
focused on the \emph{classical} R\'enyi entropy~\cite{jaconis-2013,stephan-2014}. Given 
a bipartition of the system into two complementary subregions $A$ and $B$, the classical 
R\'enyi entropy $S_n(A)$ (with $n\in\mathbb{N}$) is defined as 
%
\begin{equation}
S_n(A)\equiv \frac{1}{1-n}\log\Big(\sum\limits_{i_A\in{\mathcal C}_A} p^n_{i_A}
\Big)
\label{renyi-intro}
\end{equation}
%
Here ${\mathcal C}_A$ denotes the set of all the possible spin configurations in part 
$A$, whereas $p_{i_A}$ is the probability of the configuration $i_A$. 
Alternatively, $S_n(A)$ can be obtained from the partition function of the model on 
an \emph{ad hoc} defined ``booklet'' geometry (see section~\ref{booklet} for its 
definition). This consists of $n$ independent and identical replicas (the 
booklet ``sheets'') of the model. These \emph{physical} replicas are different 
from the \emph{fictitious} ones used to perform the disorder average. Each sheet is 
divided into two parts $A$ and $B$, containing $N_A$ and $N_B$ spins, respectively. 
The spins in part $A$ of different sheets are constrained to be equal. It is 
convenient to introduce the booklet aspect ratio $0\le\omega\le1$ as  
%
\begin{equation}
\label{a-ratio}
\omega\equiv \frac{N_A}{N}.
\end{equation}
%
Eq.~\eqref{renyi-intro} can also be used for quantum systems by interpreting $p_{i_A}$ as 
the probability of finding part $A$ in the \emph{quantum} configuration $i_A$. In particular, 
for $n=1$ Eq.~\eqref{renyi-intro} defines the subsystem Shannon entropy~\cite{alcaraz-2013,
stephan-2014-a}. From $S_n(A)$, one defines the classical mutual information 
${\mathcal I}_n(A,B)$ as 
%
\begin{equation}
{\mathcal I}_n(A,B)\equiv S_n(A)+S_n(B)-S_n(A\cup B). 
\end{equation}
%
For local models ${\mathcal I}_n$ obeys the area law~\cite{wolf-2008} ${\mathcal I}_n(A)
\propto\ell$, with $\ell$ the length of the boundary between $A$ and $B$. Remarkably, 
for different $\ell$, the ratio ${\mathcal I}_n/\ell$ exhibits a crossing at a second 
order phase transition~\cite{jaconis-2013}, implying that it can be used as a diagnostic 
tool for critical behaviors. For conformally invariant critical models more universal 
information can be extracted from the area-law corrections of ${\mathcal I}_n$~\cite{stephan-2014}. 


Although recently the study of the interplay between disorder and entanglement became a fruitful 
research area~\cite{refael-2009}, the behavior of entanglement-related quantities 
in glassy phases, and at glassy critical points, has not been explored yet (see, however,  
Ref.~\onlinecite{castelnovo-2010} for some interesting results). Here we investigate 
both the classical R\'enyi entropy $S_n(A)$ and the mutual information ${\mathcal I}_n$ 
in the S-K model, using classical Monte Carlo simulations and analytical tools. We often 
restrict ourselves to the case with $n=2$, as this is where numerical simulations are 
most efficient. As usual in disordered system, we focus on disorder-averaged quantities, 
considering $[S_n]$ and $[{\mathcal I}_n]$, with the brackets $[\cdot]$ denoting the 
average over different realizations of $J_{ij}$ (cf. Eq.~\eqref{SK-intro}).

We start discussing the thermodynamic phase diagram of the S-K model on the $n$-sheets 
booklet, as a function of temperature, and the aspect ratio $\omega$ (cf. Eq.~\eqref{a-ratio}). 
We show that for any fixed $\omega$ the S-K model exhibits a low-temperature glassy phase, 
which is divided by the standard paramagnetic one at high temperatures by a phase transition. 
Surprisingly, the critical temperature exhibits a non trivial dependence on $\omega$ 
that we are able to determine analytically. 
In the high-temperature region the permutation symmetry among the replicas, both the 
physical and the fictitious ones, is preserved. This allows us to provide an exact analytic 
expression for the free energy of the model and several derived quantities, such as the internal 
energy. We compare our results with Monte Carlo simulations, finding perfect agreement. 
Oppositely, we provide evidence that in the low-temperature phase the replica symmetry is  
broken. For instance, we numerically observe that the replica-symmetric (RS) result for the 
internal energy is systematically lower than the Monte Carlo data, as in the standard 
S-K model~\cite{sherrington-1978-prl,sherrington-1978}. This discrepancy becomes larger 
upon lowering the temperature.  

%%%%%%%%%%%%%%%%%%%%%%%%%%%%%%%%%e
\begin{figure}[t]
\includegraphics*[width=0.93\linewidth]{./draft_figs/cartoon}
\caption{ The booklet geometry considered in this work. (a) The single sheet 
 (``page'') of the booklet: The $N$ spins living on the sheet are divided into 
 two groups $A$ and $B$, containing $N_A$ and $N_B$ spins, respectively. Here 
 $\omega\equiv N_A/N$ is the booklet ratio. (b) The $n$ sheets are glued together 
 to form the booklet. The spins in part $B$ of the booklet pages are at inverse 
 temperature $\beta$. The spins in part $A$ are identified  (see Eq.~\eqref{book-constraint}). 
 As a consequence, the effective temperature in part $A$ is $n\beta$. 
 $Z(A,n,\beta)$ denotes the partition function of the S-K model on 
 the booklet. 
}
\label{cartoon}
\end{figure}
%%%%%%%%%%%%%%%%%%%%%%%%%%%%%%%%%%

Inspired by the Parisi scheme~\cite{parisi-1979}, we devise a systematic way of breaking 
the replica symmetry in successive steps. In our scheme we break only the symmetry among 
the fictitious replicas, preserving that among the physical ones. Although this appears 
natural, we are not able to provide a rigorous proof that this is the correct symmetry 
breaking pattern. In particular, we cannot exclude that the symmetry among the physical 
replicas has to be broken. As a consequence, our scheme should be regarded as an 
approximation, and not as an exact solution. We restrict ourselves to the 
one-level replica symmetry breaking ($1$-RSB). 
Surprisingly, we observe that the $1$-RSB result for the internal energy is in 
excellent agreement with the Monte Carlo data for  $\beta\lesssim 3$, whereas the RS 
approximation fails already at $\beta\approx 1$. This suggests that the $1$-RSB ansatz 
captures correctly some aspects of the replica symmetry breaking. 

Clear signatures of the replica symmetry breaking are observed in the behavior of $[S_2]$. 
First, since $[S_2]$ exhibits the volume law behavior $[S_2]\propto N$, 
we consider its density $[S_2]/N$. For finite-size systems, and for any $\omega$, $[S_2]/N$ 
exhibits a maximum at infinite temperature, and it is a decreasing function of the temperature, 
as expected. Finite-size corrections are negligible at high temperatures, whereas they increase 
upon lowering the temperature. In the paramagnetic phase we are able to determine the functional 
form of $[S_2]/N$ in the thermodynamic limit, using the replica-symmetric approximation. This 
perfectly matches the Monte Carlo data. At low temperatures deviations from the RS result are 
observed, reflecting the replica symmetry breaking. Remarkably, we observe that the one-step 
replica symmetry breaking ($1$-RSB) result $[S_2^{1\textrm{-}RSB}]/N$ is in full agreement with 
the Monte Carlo data for $\beta\lesssim 3$, confirming what is observed for the internal 
energy. 

Finally, we consider the R\'enyi mutual information $[{\mathcal I}_2]$. This obeys a volume 
law for any $\beta$ and $\omega$, in contrast with local spin models, where an area law is 
observed~\cite{wolf-2008}. The corresponding density $[{\mathcal I}_2]/N$ vanishes in 
both the infinite-temperature and the zero-temperature limits. Surprisingly, $[{\mathcal I}_2]/N$ 
does not exhibit any crossing for different system sizes at the paramagnetic-glassy transition.
This is in striking contrast with local spin models~\cite{jaconis-2013}. Interestingly, for 
any $\omega$, ${\mathcal I}_2/N$ exhibits a maximum for $\beta\approx 1$. However, the 
position of this maximum is not simply related to the paramagnetic-glassy transition. Finally, 
at high temperature $[{\mathcal I}_2]/N$ is described analytically by the RS result 
$[{\mathcal I}_2]^{RS}/N$, whereas at low temperatures one has to include the effects of the 
replica symmetry breaking. Similar to $[S_2]$, we observe that the $1$-RSB approximation 
$[{\mathcal I}^{1\textrm{-}RSB}_2]/N$, is in good agreement with the Monte Carlo data 
for $\beta\lesssim 3$.

The Article is organized as follows. In section~\ref{booklet} we introduce the classical 
R\'enyi entropy and the mutual information, reviewing their representation in terms of 
the booklet partition functions. In section~\ref{the-model} we present the structure of 
the solution of the S-K model on the booklet. Specifically, we discuss the replica 
trick, which is used to perform disorder averages, and the saddle point approximation in 
the thermodynamic limit. Section~\ref{solution} is concerned with the RS approximation. 
Precisely, in subsection~\ref{para-section} we focus on the high-temperature 
region, where this approximation becomes exact. In subsection~\ref{rs-section} we discuss 
the structure of the RS ansatz in the low-temperature region. Section~\ref{rsb-1-section} 
is devoted to   the $1$-RSB approximation. In section~\ref{mc-results} we check the validity 
of both the RS and the $1$-RSB results comparing with Monte Carlo simulations for the 
internal energy. Section~\ref{Renyi-section} and section~\ref{I2-section} discuss the classical R\'enyi 
entropy and the mutual information, respectively. Finally, we conclude in section~\ref{conclusions}.



%########################################################################
\section{The booklet construction \& and the classical R\'enyi entropy}
\label{booklet}

Given a generic \emph{classical} spin model at inverse temperature $\beta$, 
the classical R\'enyi entropy $S_n(A)$ is defined as~\cite{jaconis-2013,stephan-2014} 
%
\begin{equation}
\label{renyi}
S_n(A)\equiv \frac{1}{1-n}\log\left(\frac{Z(A,n,\beta)}{Z^n(\beta)}\right).
\end{equation}
%
Here $Z(A,n,\beta)$ is the partition function of the model on the $n$-sheet booklet, 
whereas $Z(\beta)$ is the partition functions on the plane at inverse temperatures 
$\beta$. The booklet geometry is illustrated in Fig.~\ref{cartoon}, and consists of 
$n$ identical copies (``sheets'') of the system. Each sheet is divided into two 
parts $A$ and $B$ (cf. Fig.~\ref{cartoon} (a)). The spins in part $A$ of the different 
sheets are identified (cf. Fig.~\ref{cartoon} (b)). While spins in parts $B$ of the booklet 
are at inverse temperature $\beta$, the ones in $A$ are at the effective temperature $n\beta$. 
Notice that a similar geometric construction~\cite{guerra-2002} plays an important role in the 
mathematical proof of the Parisi ansatz. 

Clearly, one has $S_n(A)\equiv1/(n-1)\log(F(A,n,\beta))-n\log(F(\beta))$, where $F(A,n,\beta)
\equiv\log(Z(A,n\beta))$ and $F(\beta)\equiv\log(Z(\beta))$ are related (apart from a factor 
$\beta$) to the free energy of the model on the booklet and on the plane, respectively.
The mutual information ${\mathcal I}_n(A,B,\beta)$ is defined~\cite{jaconis-2013,stephan-2014} 
as 
%
\begin{equation}
\label{MI}
{\mathcal I}_n(A,B,\beta)\equiv\frac{1}{1-n}\log\Big(
\frac{Z(A,n,\beta)Z(B,n,\beta)}{Z^n(\beta)Z(n\beta)}
\Big),
\end{equation}
%
where $Z(B,n,\beta)$ is obtained from $Z(A,n,\beta)$ by replacing $A$ with $B$. 
Notice that the disorder-averaged mutual information $[{\mathcal I}_n]$ and the R\'enyi 
entropy $[S_n]$ are directly related to the so-called quenched-averaged free energy 
$[F(A,n,\beta)]$, which is the main quantity of interest in disordered systems~\cite{cardy-book}. 

For clean (i.e., without disorder) \emph{local} spin models ${\mathcal I}_n$ 
obeys the boundary law 
%
\begin{equation}
{\mathcal I}_n(A,B,\beta)=\alpha_n\ell+{\mathcal G}_n+\gamma_n,
\label{GMI}
\end{equation}
%
with $\ell$ the length of the boundary between $A$ and $B$, and 
$\alpha_n,\gamma_n$ two non-universal constants. Here ${\mathcal G}_n$ 
is the so-called geometric mutual information~\cite{stephan-2014}. 
Interestingly, for critical systems ${\mathcal G}_n$ depends only on the 
geometry of $A$ and $B$, and it is universal. In particular, for conformally 
invariant models ${\mathcal G}_n$ can be calculated using standard methods of 
conformal field theory (CFT), and it allows to numerically 
extract universal information about the CFT, such as the central 
charge~\cite{stephan-2014}.  


%########################################################################
\section{The Sherrington-Kirkpatrick (S-K) model on the booklet}
\label{the-model}

Here we introduce the Sherrington-Kirkpatrick (S-K) model on the booklet. 
In subsection~\ref{the-model-def}  we define the 
model and its partition function. In subsection~\ref{replica-sec} 
we discuss the replicated booklet construction that is used to calculate the 
disorder-averaged free energy $[F(\omega,n,\beta)]$. 
In section~\ref{saddle-point} we consider the thermodynamic limit, 
using the saddle point approximation. We also introduce the overlap tensor,  
which contains all the information about the thermodynamic behavior of the 
model. The analytical formula for the replicated partition function (Eq.~\eqref{Z-ac}), 
and the saddle point equations (Eqs.~\eqref{saddle-final}\eqref{saddle-final-a}) 
for the overlap tensor are the main results of this section. 

%########################################################################
\subsection{The model and its partition function}
\label{the-model-def}

The Sherrington-Kirkpatrick (S-K) model~\cite{sherrington-1978-prl,
sherrington-1978} on the $n$-sheets booklet (cf. Fig.~\ref{cartoon}) is 
defined by the Hamiltonian
%
\begin{equation}
{\mathcal H}=-\sum\limits_{r=1}^n\left\{\sum\limits_{i<j}J_{ij}S^{(r)}_i 
S^{(r)}_j-h\sum\limits_{i=1}^NS^{(r)}_i\right\}.
\label{SK-ham}
\end{equation}
%
Here $S_i^{(r)}=\pm 1$ are classical Ising spins, $r\in[1,n]$ labels the 
different sheets (``pages'') of the booklet, $i\in[1,N]$ denotes the sites on 
each sheet, $J_{ij}$ is the interaction strength, and $h$ is an external 
magnetic field. Here we choose $h=0$. The total number of spins in the booklet 
is $nN$. The first sum inside the brackets in Eq.~\eqref{SK-ham} is over all the 
$N(N-1)/2$ pairs of spins in each sheet. Spins on different sheets do not interact. 
In each ``sheet'' of the booklet all the sites are divided into two groups $A\equiv\{1,
\dots, N_A\}$ and $B\equiv\{N_A+1,\dots,N\}$, containing $N_A$ and $N_B\equiv N-N_A$ 
sites, respectively. The spins living in part $A$ and different sheets are identified, 
i.e., $\forall i\in A$ one has  
%
\begin{equation}
S_i^{(r)}=S_{i}^{(r')}\quad\forall\, r,r'. 
\label{book-constraint}
\end{equation}
%
Since in each sheet all spins interact with each other, there is no notion of distance 
between different spins. Thus, physical observables should depend on the booklet geometry 
only through the ratio $\omega$ (cf. Eq.~\eqref{a-ratio}). 

In Eq.~\eqref{SK-ham} the couplings $J_{ij}\in\mathbb{R}$ are uncorrelated (from site to site) 
quenched random variables. $J_{ij}$ are the same in all the sheets of the booklet. Specifically, 
here $J_{ij}$ are drawn from the gaussian distribution   
%
\begin{equation}
P(J_{ij})=
\left(\frac{N}{2\pi}\right)^{1/2}
\exp\Big\{-\frac{N}{2}
\Big(J_{ij}-\frac{J_0}{N}\Big)^2\Big\}.
\label{quenched-distr}
\end{equation}
%
The mean and the variance of $P(\{J_{ij}\})$ are given as $[J_{ij}]=J_0/N$ 
and $[(J_{ij}-[J_{ij}])^2]=J^2/N$, respectively. The square brackets $[\cdot]$ denote 
the average over different realizations of $J_{ij}$.  Here we restrict ourselves to 
$J_0=0$ and $J=1$. The factors $N$ in Eq.~\eqref{quenched-distr} ensure a well-defined 
thermodynamic limit. 

The partition function $Z(\omega,n,\beta,\{J\})$ of the S-K model on the booklet 
at inverse temperature $\beta\equiv 1/T$, and for fixed disorder realization 
$\{J_{ij}\}$, reads 
%
\begin{multline}
Z(\omega,n,\beta,\{J\})\equiv\textrm{Tr}'\exp(-\beta{\mathcal H})=\\
\textrm{Tr}'\exp\Big\{\beta\sum\limits_{r}\Big(
\sum\limits_{i<j}S^{(r)}_i 
S^{(r)}_j-h\sum\limits_{i=1}^NS^{(r)}_i\Big)\Big\},
\label{book-pf}
\end{multline}
%
where $\textrm{Tr}'\equiv \sum_{\{S_i\}}$ denotes the sum over all possible 
spin configurations. The prime in $\textrm{Tr}'$ stresses that only spin configurations 
satisfying the constraint in Eq.~\eqref{book-constraint} are considered. In the two limits 
$\omega=0$ and $\omega=1$ one recovers the standard S-K model. In particular, for $\omega=0$, 
i.e., $n$ disconnected sheets, one has $Z(0,n,\beta,\{J\})=Z(\beta,\{J\})^n$, with 
$Z(\beta,\{J\})$ the partition function of the S-K model on the plane. On the other hand, 
for $\omega=1$ it is $Z(1,n,\beta,\{J\})=Z(n\beta,\{J\})$, i.e., the partition function 
of the S-K model at inverse temperature $n\beta$. The quenched averaged free energy 
$[F(\omega,n,\beta)]$, is defined as 
%
\begin{equation}
\label{free-energy}
[F(\omega,n,\beta)]\equiv -\frac{1}{\beta}\int {\mathcal D}\{J\}
\log Z(\omega,n,\beta,\{J\}), 
\end{equation}
%
where $\int{\mathcal D}\{J\}\equiv\prod\nolimits_{i<j}\int_{-\infty}^{
+\infty}dJ_{ij}P(J_{ij})$. 

At $\omega=0$ and $\omega=1$ the phase diagram of the S-K model in the 
thermodynamic limit is well established~\cite{nishimori-book}. In particular, 
at $\omega=0$ the model exhibits a standard paramagnetic phase in the high 
temperature region, whereas at low temperatures a glassy phase is present, 
with replica-symmetry breaking~\cite{nishimori-book}. The two phases are 
divided by a second order phase transition at $\beta=\beta_c=1$. The phase 
diagram for $\omega=1$ is the same, apart from the trivial redefinition
$\beta\to n\beta$. We anticipate here (see section~\ref{rs-section}) that 
a similar scenario holds for generic $\omega$. Specifically, the glassy 
replica-symmetry-broken phase at low temperatures survives for generic 
$0<\omega<1$, while at high temperature the model is paramagnetic. The 
critical point at $\beta=\beta_c(\omega)$, which marks the transition 
between the two phases, is a nontrivial function of the booklet ratio 
$\omega$ (see section~\ref{tc-section}). 

%########################################################################
\subsection{The replicated booklet and the overlap tensor}
\label{replica-sec}

The disorder-averaged free energy $[F(\omega,n,\beta)]$ (cf. 
Eq.~\eqref{free-energy}) is obtained, using the standard replica 
trick~\cite{cardy-book}, as  
%
\begin{equation}
\label{replica-trick}
[F(\omega,n,\beta)]=\lim_{\alpha\to 0}\frac{[Z^\alpha(\omega,n,\beta)]-1}
{\alpha}. 
\end{equation}
%
Here $[Z^\alpha(\omega,n,\beta)]$ is the disorder-averaged partition 
function of $\alpha\in\mathbb{N}$ independent copies of the S-K model on the 
booklet. Precisely, $[Z^\alpha(\omega,n,\beta)]$ reads 
%
\begin{multline}
[Z^\alpha(\omega,n,\beta)]=\int{\mathcal D}\{J\}
\tr'\exp\sum\limits_{r,\gamma}
\Big\{\\
\beta\sum\limits_{i<j}J_{ij}S_i^{(r,\gamma)}S_j^{(r,\gamma)}
+\beta h\sum\limits_{i}S^{(r,\gamma)}_i\Big\},
\label{rep-Z}
\end{multline}
%
where the index $\gamma=1,2,\dots,\alpha$ labels the different fictitious 
replicas introduced in Eq.~\eqref{replica-trick}, whereas $r$ denote the 
physical replicas, as in Eq.~\eqref{book-pf}. Clearly, physical and fictitious 
replicas do not interact. Notice that $Z^{\alpha}(\omega,n,\beta)$ can be 
thought of as the partition function of the S-K model on a ``replicated'' 
booklet. 

Using Eq.~\eqref{quenched-distr}, the disorder average in Eq.~\eqref{rep-Z} 
can be performed explicitly, to obtain 
%
\begin{multline}
[Z^\alpha(\omega,n,\beta)]=\tr'\exp\sum\limits_{r,\gamma}\Big\{\\
\frac{1}{N}
\sum\limits_{i<j}\Big(
\frac{\beta^2}{2}\sum\limits_{\gamma',r'}S^{(r,\gamma)}_iS^{(r,\gamma)}_j
S^{(r',\gamma')}_iS^{(r',\gamma')}_j\\
+\beta J_0 S_i^{(r,\gamma)}S_j^{(r,\gamma)}\Big)
+\beta h\sum\limits_{i}
S_i^{(r,\gamma)}\Big\}.
\label{dis-aver-Z}
\end{multline}
%
In contrast with Eq.~\eqref{rep-Z}, physical and unphysical replicas are 
now coupled by a four-spin interaction. It is convenient to introduce the  
Hubbard-Stratonovich variables $q_{\gamma\gamma'}^{rr'}$ and $m_\gamma^r$. 
Following the spin glass literature~\cite{parisi-book}, we dub $q_{\gamma
\gamma'}^{rr'}$ the overlap tensor. In the standard S-K model (i.e., for $n=1$) 
$q_{\gamma\gamma'}^{rr'}$ becomes a $\alpha\times\alpha$ 
matrix~\cite{sherrington-1978-prl}. Eq.~\eqref{dis-aver-Z} now 
yields  
%
\begin{multline}
\label{hs-Z}
[Z^\alpha(\omega,n,\beta)]=\exp\Big(\frac{\beta^2Nn\alpha}{4}\Big)
\int\prod_{\substack{\gamma\le\gamma'\\r,r'}}
dq^{rr'}_{\gamma\gamma'}
\int\prod_{\gamma,r}dm_\gamma^r\\
\tr'\exp
\Big\{-N {\mathcal K}(\{q,m\})
+\sum_i{\mathcal L}_i(\{q,m\})\Big\},
\end{multline}
%
where we neglected subleading contributions ${\mathcal O}(1/N)$ in 
the thermodynamic limit. Here ${\mathcal K}(\{q,m\})$ is spin-independent 
and it reads 
%
\begin{multline}
{\mathcal K}(\{q,m\})\equiv 
\frac{\beta^2}{2}\Big(\sum\limits_{\gamma<\gamma'}\sum
\limits_{r,r'} (q_{\gamma\gamma'}^{rr'})^2
\\
+\sum\limits_{\gamma}\sum\limits_{r<r'}(q_{\gamma\gamma}^{
rr'})^2\Big)
+\frac{\beta}{2 J_0}\sum\limits_{\gamma r}(m_\gamma^r)^2. 
\label{Gamma}
\end{multline}
%
On the other hand ${\mathcal L}_i(\{q,m\})$ depends on the spin degrees of 
freedom, and it is given as   
%
\begin{multline}
{\mathcal L}_i(\{q,m\})
\equiv\beta^2\sum\limits_{\gamma<\gamma'}\sum\limits_{r,r'}
q_{\gamma\gamma'}^{rr'}S_i^{(r,\gamma)}
S_i^{(r',\gamma')}+\\
\beta^2\sum_\gamma\sum\limits_{r<r'}
q_{\gamma\gamma}^{rr'}S^{(r,\gamma)}_i
S^{(r',\gamma)}_i
+\beta\sum\limits_{\gamma r}(m_\gamma^r+h)
S_i^{(r,\gamma)}.
\label{mf-action}
\end{multline}
%
Interestingly, ${\mathcal L}_i$ describes a system of $n\alpha$ spins living 
in the replica space with the long-range interaction $q_{\gamma\gamma'}^{rr'}$, 
and a magnetic field $m_\gamma^r+h$. Notice that, while the first term in 
Eq.~\eqref{mf-action} is off-diagonal in the space of the fictitious replicas, 
the second one is diagonal. We anticipate here that the latter fully determines the 
behavior of the model in the paramagnetic phase (see section~\ref{para-section}). 

Since in Eq.~\eqref{hs-Z} spins on  different sites are decoupled, one can 
perform the trace over the spins in parts $A$ and $B$ (see Fig.~\ref{cartoon}) 
independently, to obtain 
%
\begin{multline}
[Z^\alpha(\omega,n,\beta)]=
\int\prod_{\substack{\gamma\le\gamma'\\r,r'}}
dq^{rr'}_{\gamma\gamma'}
\int\prod_{\gamma,r}dm_\gamma^r
\exp\Big\{N\Big(\\
\frac{\beta^2n\alpha}{4}+\omega\log\tr_Ae^{{\mathcal L}}+
(1-\omega)\log\tr_Be^{{\mathcal L}}-{\mathcal K}\Big)\Big\}.
\label{z-final}
\end{multline}
%
Here to lighten the notation we drop the dependence on the coordinate $i$ 
and the arguments of ${\mathcal L}_i(\{q,m\})$ and ${\mathcal K}(\{q,m\})$. 
$\textrm{Tr}_{A}$ and $\textrm{Tr}_{B}$  denote the trace over the spin degrees 
of freedom living in parts $A$ and $B$ of the booklet.
The subscript $A$ in $\tr_A$ is to stress that spins living in different physical 
replicas (i.e., for $r\ne r'$ in Eq.~\eqref{mf-action}) are identified (due 
to the booklet constraint in Eq.~\eqref{book-constraint}), whereas they have 
to be treated as independent variables in performing $\tr_B$.  

%########################################################################
\subsection{The saddle point approximation}
\label{saddle-point}

In the thermodynamic limit, i.e., for $N,N_A\to\infty$, at fixed ratio 
$\omega\equiv N_A/N$, one can take the saddle point approximation in 
Eq.~\eqref{z-final}, which yields   
%
\begin{multline}
[Z^\alpha(\omega,n,\beta)]\approx
\exp\Big\{N\alpha\Big(\frac{\beta^2n}{4}-\frac{\mathcal K}{\alpha}\\
+\frac{\omega}{\alpha}\log\tr_A\exp({\mathcal L})+
\frac{1}{\alpha}(1-\omega)\log\tr_B\exp({\mathcal L})
\Big)\Big\}. 
\label{Z-ac}
\end{multline}
%
The overlap tensor $q_{\gamma\gamma'}^{rr'}$ and $m_\gamma^r$ are 
determined by solving the saddle point equations 
%
\begin{align}
\label{saddle}
& \frac{\partial}{\partial q_{\gamma\gamma'}^{rr'}}
\left(\omega\log\tr_A e^{{\mathcal L}}+(1-\omega)\log\tr_B 
e^{{\mathcal L}}\right)=q_{\gamma\gamma'}^{rr'}\\
%\nonumber
\label{saddle-a}
& \frac{\partial}{\partial m_\gamma^r}\left(\omega\log\tr_A 
e^{{\mathcal L}}+(1-\omega)\log\tr_B e^{{\mathcal L}}\right)=
\frac{1}{J_0}m_\gamma^r.
\end{align}
%
It is enlightening to rewrite Eqs.~\eqref{saddle}~\eqref{saddle-a} as 
%
\begin{align}
\label{saddle-final}
& q_{\gamma\gamma'}^{rr'}=
\omega\langle S^{(r,\gamma)} S^{(r',\gamma')}\rangle_A+
(1-\omega)\langle S^{(r,\gamma)}S^{(r',\gamma')}
\rangle_B\\
& \frac{1}{J_0}m_\gamma^r=\omega\langle S^{(r,\gamma)}
\rangle_A+(1-\omega)\langle S^{(r,\gamma)}
\rangle_B
\label{saddle-final-a}
\end{align}
%
where $\langle {\mathcal O}\rangle_{A(B)}\equiv (Z_{A(B)})^{-1}
\tr_{A(B)}\{{\mathcal O}\exp({\mathcal L})\}$ with $Z_{A(B)}\equiv\tr_{A(B)}
\exp({\mathcal L})$. Notice that Eq.~\eqref{saddle-final} implies 
that $q_{\gamma\gamma}^{rr}=1$ $\forall\gamma,r$. For $\omega=0$ and $n=1$, 
one recovers the saddle point equations for the standard SK model~\cite{parisi-book,
nishimori-book}. 

In order to calculate the free energy $[F(\omega,n,\beta)]$ one has to solve 
Eqs.~\eqref{saddle-final}~\eqref{saddle-final-a}, take the analytic continuation 
$\alpha\in\mathbb{R}$, and, finally, the limit $\alpha\to 0$.  
(cf. Eq.~\eqref{replica-trick}). Although it is possible to solve 
Eqs.~\eqref{saddle-final}~\eqref{saddle-final-a} numerically for any fixed $r,
\alpha\in\mathbb{N}$, taking the analytic continuation $\alpha\in\mathbb{R}$ is a 
formidable task, since the dependence of $Z^\alpha(\omega,n,\beta)$ on $\alpha$ is in 
general too complicated. The strategy is usually to choose a specific 
form of the overlap tensor $q_{\gamma\gamma'}^{rr'}$ in terms of ``few'' 
parameters, which allows to perform the analytic continuation and the limit 
$\alpha\to 0$ exactly. 

For the standard S-K model (i.e., for $n=1$) the simplest parametrization is the 
replica-symmetric one (RS), which amounts to taking $q^{11}_{\gamma\gamma'}=q$. 
This relies on the observation that the fictitious replicas appear symmetrically 
in Eq.~\eqref{rep-Z}. Although the RS ansatz is correct at high temperatures, it 
fails in the glassy phase at low temperatures, where the permutation invariance 
within the replicas has to be broken~\cite{almeida-1978}. The celebrated Parisi 
ansatz~\cite{parisi-1979} provides a systematic scheme to break the replica 
symmetry in successive steps, and it allows to capture the glassy behavior of 
the S-K model at low temperature. We anticipate (see section~\ref{rsb-1-section} for 
the details) that a similar scheme has to be used to describe the glassy phase of 
the S-K model on the booklet. 


%########################################################################
\section{The replica-symmetric (RS) ansatz}
\label{solution}

In this section we present the solution of the S-K model on the booklet, using 
the replica symmetric (RS) approximation. In subsection~\ref{para-section} we focus 
on the high temperature phase, where this approximation is exact, and the 
behavior of the model is fully determined by the diagonal part of the overlap 
tensor $q_{\gamma\gamma'}^{rr'}$ (see section~\ref{replica-sec} for its 
definition). The generic structure of $q_{\gamma\gamma'}^{rr'}$ 
within the RS ansatz  is discussed in subsection~\ref{rs-section}. This allows us 
to determine the critical temperature of the paramagnetic-glassy transition. 

%########################################################################
\subsection{The paramagnetic phase}
\label{para-section}

%%%%%%%%%%%%%%%%%%%%%%%%%%%%%%%%%e
\begin{figure}[t]
\includegraphics*[width=0.9\linewidth]{./draft_figs/RSB0_saddle_ht}
\caption{
 The S-K model on the $2$-sheets booklet with zero magnetic field in the 
 paramagnetic phase: the solution $q_0$ of the saddle point 
 equation~\eqref{saddle-high-t} plotted as a function of the booklet aspect 
 ratio $\omega$ and several values of the inverse temperature $\beta=1/T$. 
 At $\omega=1$ one has $q_0=1,\forall\beta$. In the zero-temperature limit 
 one has $q_0\to 1,\forall\omega$. The straight line is the infinite 
 temperature result. 
}
\label{RSB0_saddle_ht}
\end{figure}
%%%%%%%%%%%%%%%%%%%%%%%%%%%%%%%%%%

Here we provide the exact analytical expression for the disorder-averaged free 
energy $[F_{para}(\omega,n,\beta)]$  in the paramagnetic non-glassy phase. 
We start discussing the infinite temperature limit (i.e., $\beta\to 0$), 
restricting ourselves to zero magnetic field. 
Using Eq.~\eqref{mf-action}, a standard high-temperature expansion yields 
$\tr_Ae^{{\mathcal L}}=2^{\alpha}+{\mathcal O}(\beta^2)$ and $\tr_B e^{{
\mathcal L}}=2^{n\alpha}+{\mathcal O}(\beta^2)$, implying  
%
\begin{align}
& \langle S^{(r,\gamma)}S^{(r',\gamma')}\rangle_B=2^{n\alpha}
\delta_{\gamma,\gamma'}\delta_{r,r'}+{\mathcal O}(\beta^2)\\
& \langle S^{(r,\gamma)}S^{(r',\gamma')}\rangle_A=2^{\alpha}
\delta_{\gamma,\gamma'}+{\mathcal O}(\beta^2).
\end{align}
%
Using Eq.~\eqref{saddle-final}, it is straightforward to obtain the 
infinite-temperature overlap tensor $q_{\gamma\gamma'}^{rr'}$ as 
%
\begin{equation}
q_{\gamma\gamma'}^{rr'}=(1-\omega)\delta_{\gamma,\gamma'}\delta_{r,r'}+
\omega\delta_{\gamma,\gamma'}.
\label{inf-t-q}
\end{equation}
%
Remarkably, $q_{\gamma\gamma'}^{rr'}$ becomes diagonal in both the 
indices $\gamma,\gamma'$ and $r,r'$, i.e., the physical and fictitious 
replica spaces. Using Eq.~\eqref{Z-ac} and Eq.~\eqref{replica-trick}, 
after performing the analytic continuation $\alpha\to 0$, one obtains  
%
\begin{multline}
[F_{para}(\omega,n,\beta)]=N\Big\{(n-\omega(n-1))\log(2)\\
+\frac{\beta^2}{4}(\omega^2(n^2-n)+n)\Big\}+
{\mathcal O}(\beta^4).
\label{logZ-ht}
\end{multline}
% 
It is natural to expect that for arbitrary $\beta\le\beta_c$, with 
$\beta_c$ the critical temperature of the paramagnetic-glassy 
transition, the overlap tensor $q_{\gamma\gamma'}^{rr'}$ remains 
diagonal. This suggests the ansatz 
%
\begin{equation}
q_{\gamma\gamma'}^{rr'}=(1-q_0)\delta_{r,r'}\delta_{\gamma,\gamma'}+
q_0\delta_{\gamma,\gamma'}.
\label{high-t-q}
\end{equation}
%
with $q_0\in {\mathbb R}$ a parameter. The ansatz~\eqref{high-t-q} is formally 
obtained from Eq.~\eqref{inf-t-q} by replacing $\omega\to q_0$. Notice that 
one has $q_{\gamma\gamma}^{rr}=1$, in agreement with Eqs.~\eqref{saddle-final}. 
Using Eq.~\eqref{high-t-q}, one obtains ${\mathcal L}_{para}$ (cf. 
Eq.~\eqref{mf-action}) as   
%
\begin{equation}
\label{high-t-action}
{\mathcal L}_{para}=\beta^2\frac{q_0}{2}\Big\{\sum_\gamma\sum_{rr'}S^{(r,
\gamma)}S^{(r',\gamma)}-n\alpha\Big\}.
\end{equation}
%
After introducing the Hubbard-Stratonovich variables $z_\lambda$ (with 
$\lambda=1,\dots,\alpha$), one can write 
%
\begin{multline}
\label{ht-saddle}
\textrm{Tr}_B\exp({\mathcal L}_{para})=\\\tr_B\int\prod_\lambda Dz_{\lambda}
\exp\Big(z_\lambda\beta\sqrt{q_0}\sum_r S^{(r,\lambda)}\Big),
\end{multline}
%
where $\int Dz f(z)\equiv(2\pi)^{-1/2}\int\exp(-z^2/2)f(z)$. Notice that 
due to the square root in Eq.~\eqref{ht-saddle}, one has the constraint 
$q_0>0$. Moreover, from Eq.~\eqref{high-t-action} 
one obtains $\textrm{Tr}_A\exp({\mathcal L})=2^{n\alpha}$. The trace 
$\textrm{Tr}_B$ in Eq.~\eqref{ht-saddle} can be performed explicitly. 
Using Eq.~\eqref{Z-ac} and Eq.~\eqref{replica-trick}, one obtains 
the free energy $[F_{para}(\omega,n,\beta)]$ as  
%
%%%%%%%%%%%%%%%%%%%%%%%%%%%%%%%%%e
\begin{figure}[t]
\includegraphics*[width=0.9\linewidth]{./draft_figs/RSB0_saddle}
\caption{\label{RSB0-saddle}
 The S-K model on the $2$-sheets booklet with zero magnetic field. The 
 overlap tensor in the replica-symmetric (RS) approximation (see Eq.~\eqref{rs-ansatz}): 
 the solutions $q_0$ and $q_0'$ (shown as full and dotted lines, respectively) 
 of the saddle point equations~\eqref{RS-saddle-1}\eqref{RS-saddle-2} 
 plotted as a function of the booklet aspect ratio $\omega$, and inverse 
 temperature $\beta=1/T=1,3/2,2$. At $\omega=1$ one has that $q_0=1,
 \forall\beta$. In the limit $\beta\to\infty$ it is $q_0,q_0'\to 1,\forall
 \omega$.
}
\label{RSB0_saddle}
\end{figure}
%%%%%%%%%%%%%%%%%%%%%%%%%%%%%%%%%%
%
\begin{multline}
[F_{para}(\omega,n,\beta)]=N\Big\{
n\log(2)-\omega(n-1)\log(2)\\+
\beta^2\Big(\frac{n}{4}-\frac{q_0^2}{4}(n^2-n)
+\frac{\omega}{2}q_0n^2-q_0n\Big)\\
+(1-\omega)\log\int Dz\cosh^n(z\sqrt{q_0}\beta)
\Big\},
\label{ht-logZ}
\end{multline}
%
where $q_0$ is determined by solving the saddle point condition $\partial[F_{para}
(\omega,n,\beta)]/\partial q_0=0$. For the $2$-sheets booklet (i.e., $n=2$) this is  
given as   
%
\begin{equation}
q_0=\omega+(1-\omega)\tanh(\beta^2 q_0).
\label{saddle-high-t}
\end{equation}
%
Alternatively, Eq.~\eqref{saddle-high-t} can be obtained by substituting the 
ansatz~\eqref{high-t-q} in Eqs.~\eqref{saddle-final}~\eqref{saddle-final-a}. 

Clearly, for two independent copies of the S-K model, i.e., $\omega=0$,  
Eq.~\eqref{saddle-high-t} gives $q_0=0$ for $\beta\le1$, whereas 
one has $q_0\ne0$ for $\beta>1$. On the other hand, for $\omega=1$ 
one has $q_0=1$ $\forall\beta$. For intermediate $0<\omega<1$, $q_0$ 
is plotted as a function of $\omega$ in Fig.~\ref{RSB0_saddle_ht}. For $\beta=0$ it 
is $q_0=\omega$ (straight line in the Figure). In the low-temperature limit 
one has $q_0\to 1$, for any $\omega$. In particular, it is straightforward 
to check that  $q_0\approx 1-2(1-\omega)\exp(-2\beta^2)$ for $\beta\to\infty$. 


%##########################################################################
\subsection{The replica-symmetric (RS) approximation} 
\label{rs-section}

In the replica-symmetric (RS) approximation one writes the overlap tensor  
$q_{\gamma\gamma'}^{rr'}$ as
%
\begin{equation}
q_{\gamma\gamma'}^{rr'}=(1-q_0)\delta_{r,r'}\delta_{\gamma,\gamma'}+
q_0\delta_{\gamma,\gamma'}+(1-\delta_{\gamma,\gamma'})q'_0.
\label{rs-ansatz}
\end{equation}
%
The first two terms in Eq.~\eqref{rs-ansatz} are the same as in the paramagnetic 
phase (cf. Eq.~\eqref{high-t-q}). The last term sets $q_{\gamma\gamma'}^{rr'}=q'_0$ 
$\forall\gamma\ne\gamma'$ and $\forall r,r'$. Clearly, $q_{\gamma\gamma'}^{rr'}$ is 
symmetric in both the physical and fictitious replica spaces. We do not have any 
rigorous argument to justify the ansatz~\eqref{rs-ansatz}, besides its simplicity. 
However, we numerically observe that it captures quite accurately the behavior of 
the model, at least around the paramagnetic-glassy transition (see section~\ref{mc-results} 
for the comparison with Monte Carlo data). 

Using Eq.~\eqref{rs-ansatz} and Eqs.~\eqref{replica-trick}\eqref{Z-ac} one obtains 
the replica-symmetric approximation for the free energy $[F_{RS}(\omega,n,\beta)]$ 
as 
%
\begin{multline}
[F_{RS}(\omega,n,\beta)]=\\\lim_{\alpha\to 0}\Big\{
N\Big[\frac{\beta^2}{4}n-\frac{\beta^2}{4}\Big(
(q_0^2-(q'_0)^2)n^2-q_0^2n\Big)\\
+\frac{1-\omega}{\alpha}\log\tr_B\exp({\mathcal L}_{RS})\\
+\frac{\omega}{\alpha}\log\tr_A \exp({\mathcal L}_{RS})
\Big]\Big\},
\label{rs-F}
\end{multline}
%
where ${\mathcal L}_{RS}$ is obtained by substituting Eq.~\eqref{rs-ansatz} 
in Eq.~\eqref{mf-action}, which yields 
%
\begin{multline}
\label{l-rs}
{\mathcal L}_{RS}=\frac{q_0'}{2}\beta^2
\sum\limits_{\gamma\gamma'}\sum_{rr'}
S^{(r,\gamma)}S^{(r',\gamma')}\\
+\frac{q_0-q_0'}{2}\beta^2\sum_\gamma\sum_{rr'}
S^{(r,\gamma)}S^{(r',\gamma')}-\frac{q_0}{2}\beta^2n\alpha.
\end{multline}
%
To calculate the last two terms in Eq.~\eqref{rs-F} one has to introduce 
two auxiliary Hubbard-Stratonovich variables $z,z'$, similar to the 
paramagnetic phase (cf. section~\ref{para-section}). Thus, after performing 
the trace over the spin variables, in the limit $\alpha\to0$, one obtains 
%
\begin{align}
\label{eq1}
 \log\tr_B\exp({\mathcal L}_{RS})=&
-\frac{q_0}{2}\beta^2 n\alpha+n\alpha\log(2)\\\nonumber
&+\alpha\int Dz
\log\int Dz'H_{RS}^n(z,z'),\\
\label{eq2}
 \log\tr_A\exp({\mathcal L}_{RS})=&
-\frac{q_0}{2}\beta^2 n\alpha+\alpha\log(2)\\\nonumber
&+\alpha\int Dz
\log\int Dz'H_{RS}(nz,nz'), 
\end{align}
%
where $H_{RS}(z,z')\equiv\cosh(\beta z\sqrt{q_0'}+\beta z'\sqrt{q_0-q_0'})$. 
Notice that because of the square roots in the definition of $H_{RS}(z,z')$, 
one has the constraint $0\le q_0'\le q_0\le 1$. 

In Eqs.~\eqref{eq1}\eqref{eq2} $q_0,q_0'$ satisfy the saddle point conditions 
$\partial[F_{RS}(\omega,n,\beta)]/\partial q_0=\partial[F_{RS}(\omega,n,\beta)]/
\partial q'_0=0$ (cf. Eqs.~\eqref{RS-saddle-1}\eqref{RS-saddle-2} for their 
form for $n=2$). The resulting $q_0$ and $q_0'$ are plotted in Fig.~\ref{RSB0-saddle} 
(full and dotted lines, respectively) as function of $\omega$ and for several 
values of $\beta$. Clearly, for any $\beta$ one has $q_0=1$ in the limit 
$\omega\to 1$. Also, in the zero-temperature limit $\beta\to\infty$ one has 
that $q_0\to 1$ and $q_0'\to 1$, for any $\omega$. Moreover, a simple calculation 
yields  
%
\begin{align}
\label{low-t}
& q_0=1 -(1-\omega)\sqrt{\frac{2}{\pi}}\frac{1}{\beta}\exp\Big(-\frac{1}{\pi}-
\sqrt{\frac{2}{\pi}}\beta\Big)+\dots
\\ 
& q_0'=1-\frac{1}{\sqrt{2\pi}\beta}-\frac{1}{2\pi\beta^2}+{\mathcal O}(\beta^{-3}),
\end{align}
%
with the dots denoting exponentially suppressed terms in the limit $\beta\to\infty$. 
Interestingly, from Eq.~\eqref{low-t} one has that $q_0\to 1$ exponentially in the limit $\beta\to\infty$, 
as in the paramagnetic phase (cf. section~\ref{para-section}), whereas 
$q_0'-1\propto 1/\beta$, similar to the standard SK model~\cite{nishimori-book}. 


%%%%%%%%%%%%%%%%%%%%%%%%%%%%%%%%%e
\begin{figure}[t]
\includegraphics*[width=0.9\linewidth]{./draft_figs/betac}
\caption{
 The critical temperature $\beta_c\equiv 1/T_c$ of the paramagnetic-glassy 
 transition for the S-K model on the $2$-sheets booklet (see Fig.~\ref{cartoon}) 
 with zero magnetic field: $\beta_c$ as a function of the booklet ratio $\omega
 \equiv N_A/N$. Here $\beta_c$  is obtained from the replica-symmetric (RS) 
 approximation. Notice that $\beta_c=1$ and $\beta_c=1/2$ for $\omega=0$ and 
 $\omega=1$, respectively. 
}
\label{beta_c}
\end{figure}
%%%%%%%%%%%%%%%%%%%%%%%%%%%%%%%%%%


%##########################################################################
\subsection{The paramagnetic-glassy transition}
\label{tc-section}

Using the replica-symmetric ansatz Eq.~\eqref{rs-ansatz} one can determine 
the critical temperature $\beta_c$ of the paramagnetic-glassy transition. 
Near the glassy transition one should expect $q'_0\to0$, whereas $q_0$ 
should remain finite (see section~\ref{para-section}). By expanding $[F_{RS}(\omega,n,\beta)]$ 
(cf. Eq~\eqref{rs-F}) for small $q_0'$, and keeping only terms up to ${\mathcal O}((q_0')^2)$, 
one obtains a standard Landau theory. Thus, $\beta_c$ is obtained by imposing that the 
coefficient of the quadratic term $q_0'^2$ changes sign. This leads to the equation 
%
\begin{equation}
\exp(-4q_0\beta^2_c)+2\exp(-2q_0\beta^2_c)=\frac{1-4\beta^2_c}
{4\beta^2_c\omega-1},
\label{tc}
\end{equation}
%
where $q_0$ is obtained by solving the high-temperature saddle point equation~\eqref{saddle-high-t}. 
The resulting $\beta_c$ is plotted in Fig.~\ref{beta_c} as a function of $\omega$. 

%%%%%%%%%%%%%%%%%%%%%%%%%%%%%%%%%%
\begin{figure}[t]
\includegraphics*[width=0.93\linewidth]{./draft_figs/U_w05}
\caption{
 The S-K model on the $2$-sheets booklet with zero magnetic field and 
 booklet ratio $\omega=1/2$ (see Fig.~\ref{cartoon}): The internal energy 
 per spin $U/N$ as a function of the inverse temperature $\beta$. The triangles, 
 squares, and rhombi, denote the Monte Carlo data for a booklet with 
 $N=32,64,128$ spins per sheet. The plus symbols are the extrapolations to the 
 thermodynamic limit $N\to\infty$, at fixed $\omega$. The dash-dotted line is the 
 analytical result $U_{RS}/N$ obtained using the replica-symmetric (RS) approximation. 
 The horizontal dotted line is the exact zero-temperature result $U/N\approx-1.52642$.
}
\label{U-MC}
\end{figure}
%%%%%%%%%%%%%%%%%%%%%%%%%%%%%%%%%%

%##########################################################################
\section{The one-step replica-symmetry-breaking (1-RSB) approximation}
\label{rsb-1-section}

In this section we go beyond the replica-symmetric approximation, including some 
of the effects of the replica symmetry breaking. More specifically, 
here we discuss the one-step replica symmetry breaking (1-RSB) approximation. 
The overlap tensor $q_{\gamma\gamma'}^{rr'}$ now reads  
%
\begin{equation}
\label{q-rsb1}
q_{\gamma\gamma'}^{rr'}=(1-q_0)\delta_{\gamma,\gamma'}\delta_{r,r'} +
q_0\delta_{\gamma,\gamma'}+(1-\delta_{\gamma,\gamma'})q', 
\end{equation}
%
which is formally equivalent to the RS ansatz in Eq.~\eqref{rs-ansatz}, apart from 
the trivial redefinition $q_0'\to q'$. However, in contrast with Eq.~\eqref{rs-ansatz}, 
where $q_0'\in\mathbb{R}$ is a number, here $q'$ is a matrix. Inspired by the Parisi 
scheme for the standard S-K model~\cite{parisi-1979}, we choose 
%
\begin{equation}
q'=\left\{
\begin{array}{cc}
q_1' & \textrm{if}\, \lfloor\gamma/m_1\rfloor=\lfloor\gamma'/m_1\rfloor\\\\
q_0' & \textrm{otherwise}\\
\end{array}
\right.
\label{q-rsb1a}
\end{equation}
%
where $q_0',q_1'\in{\mathbb R}$, $m_1\in{\mathbb N}$, and $\lfloor\cdot\rfloor$ 
denotes the floor function. Notice that the off-diagonal elements of $q'$ (i.e., for 
$\gamma\ne\gamma'$) do not depend on $r,r'$, meaning that, although the permutation 
symmetry between the fictitious replicas is broken, the symmetry among the physical ones 
is preserved. The choice in Eq.~\eqref{q-rsb1a} corresponds to a simple block-diagonal 
structure for $q'$: the matrix elements of the $m_1\times m_1$ diagonal blocks of $q'$ 
are set to $q_1'$, whereas all the off-diagonal elements are set to $q_0'$. As for the 
replica-symmetric ansatz in Eq.~\eqref{rs-ansatz}, we do not have any rigorous argument 
to justify Eq.~\eqref{q-rsb1a} (see section~\eqref{mc-results}, however, for numerical 
results).

The effective interaction ${\mathcal L}_{1\textrm{-}RSB}$ (cf. Eq.~\eqref{mf-action}) 
in the replica space is obtained by substituting Eq.~\eqref{q-rsb1} in 
Eq.~\eqref{mf-action}. This yields  
%
\begin{multline}
{\mathcal L}_{1\textrm{-}RSB}/\beta^2=-\frac{q_0}{2}n\alpha
-\frac{q_1'-q_0}{2}\sum\limits_{\sigma=1}^{\alpha/m_1}
\sum\limits_{\gamma\in B_\sigma}\Big(\sum_r S^{(r,\gamma)}\Big)^2\\
+\frac{q_0'}{2}\Big(\sum\limits_{\gamma,r}
S^{(r,\gamma)}\Big)^2-
\frac{q_0'-q_1'}{2}\sum\limits_{\sigma=1}^{\alpha/m_1}
\Big(\sum\limits_{\gamma\in B_\sigma,r}S^{(r,\gamma)}\Big)^2,
\label{blocks}
\end{multline}
%
where we defined $B_\sigma\equiv[\sigma m_1,(\sigma+1)m_1)$ with $\sigma\in\mathbb{N}$. 
It is convenient to introduce the Hubbard-Stratonovich variables $z,z_\sigma,z_{\sigma,\gamma}$ 
(one for each term in Eq.~\eqref{blocks}). One then obtains 
%
\begin{multline}
\label{step}
\log\tr'\exp({\mathcal L}_{1\textrm{-}RSB})=
-\frac{q_0}{2}\beta^2 n\alpha\\
+\log\tr'\int Dz\prod_\sigma\int Dz_\sigma\prod\limits_{\gamma\in B_\sigma}
\int Dz_{\sigma,\gamma}\prod_r\exp\Big\{\\
\beta\Big(z\sqrt{q'_0}+z_{\sigma,\gamma}\sqrt{q_0-q_1'}+
z_{\sigma}\sqrt{q_1'-q_0'}\Big)S^{(r,\gamma)}\Big\}.
\end{multline}
%
The trace over the spin variables in Eq.~\eqref{step} can be now performed explicitly. 
Finally, one obtains the 1-RSB approximation for the free energy 
$[F_{1\textrm{-}RSB}(\omega,n,\beta)]$ as 
%
\begin{widetext}
\begin{multline}
[F_{1\textrm{-}RSB}(\omega,n\beta)]/N=
(\omega+(1-\omega)n)\log(2)+
\frac{n}{4}\beta^2\Big(1+nq_0'^2
-n(m_1-1)(q_1'^2-q_0'^2)-
(n-1)q_0^2-2q_0\Big)\\
+\int Dz\left\{\frac{\omega}{m_1}\log\int Dz'\Big\{\int Dz''
H_{1\textrm{-}RSB}^n(z,z',z'')\Big\}^{m_1}
+\frac{1-\omega}{m_1}
\log\int Dz'\Big\{\int Dz''
H_{1\textrm{-}RSB}(nz,nz',nz'')\Big\}^{m_1}
\right\}, 
\label{RSB-1-logZ}
\end{multline}
\end{widetext}
%
where we defined $H_{1\textrm{-}RSB}(z,z',z'')$ as 
%
\begin{multline}
H_{1\textrm{-}RSB}(z,z',z'')\equiv\cosh(z\beta\sqrt{q'_0}\\+
z''\beta\sqrt{q_0-q'_1} +z'\beta\sqrt{q_1'-q_0'}).
\label{H1}
\end{multline}
%
Similar to the replica-symmetric situation (see section~\ref{rs-section}), 
from Eq.~\eqref{H1} one has the constraint $0\le q_0'\le q_1'\le q_0\le 1$.
The parameters $q_0,q_0',q_1',m_1$ are obtained by solving the saddle 
point equations~\eqref{RSB-1-saddle-a}\eqref{RSB-1-saddle-b}\eqref{RSB-1-saddle-c}
\eqref{RSB-1-saddle-d}. One should remark that, although formally $m_1$ is an 
integer, one obtains $m_1\in\mathbb{R}$ from the saddle point equations. 
Clearly, the replica-symmetric result $[F_{RS}(\omega,n,\beta)]$ (cf. 
Eq.~\eqref{rs-F}) is recovered from Eq.~\eqref{RSB-1-logZ} in the limit 
$q_1'=q_0'$, while the free energy in the paramagnetic phase $[F_{para}(\omega,n,
\beta)]$ (cf. Eq.~\eqref{ht-logZ}) corresponds to $q_1'=q_0'=0$.

%##########################################################################
\section{Monte Carlo results: The internal energy}
\label{mc-results}

In this section we numerically confirm the analytical results of section~\ref{solution}. 
We discuss Monte Carlo (MC) data for the S-K model on the $2$-sheets booklet 
with zero external magnetic field. We focus on the internal energy $U(\omega,n,
\beta)$ 
%
\begin{equation}
\label{U-def}
U(\omega,n,\beta)\equiv-\frac{\partial}{\partial\beta}
[\log(Z(\omega,n,\beta))]. 
\end{equation}
%  
Fig.~\ref{U-MC} plots the MC data for $U(\omega,2,\beta)$ versus the inverse temperature 
$\beta$, for $\omega=1/2$. The circles, squares, and triangles 
are the MC results for different sizes, i.e., number of spins per sheet, $N=32,64,128$. 
The vertical dotted line is the critical temperature $\beta_c\approx 0.6$ of 
the paramagnetic-glassy transition (cf. Fig.~\ref{beta_c}). In the high-temperature 
region finite size effects are small, and already for $N=64$ the MC data are indistinguishable 
from the thermodynamic limit. Oppositely, stronger scaling corrections are visible in 
the low-temperature phase at $\beta>\beta_c$. 
The plus symbols in Fig.~\ref{U-MC} are the numerical extrapolations in the 
thermodynamic limit. These are obtained by fitting the finite size MC data 
to the ansatz $U/N=u_{\infty}(\omega,\beta)+c/N^{\phi}$, where 
$u_{\infty}$ is energy density in the thermodynamic limit, $c$ a fitting parameter, and 
$\phi$ the exponent of the scaling corrections. In our fits we fix $\phi=2/3$, which 
is the exponent governing the finite-size corrections of $U/N$ in 
the standard S-K model~\cite{billoire-2007,aspelmeier-2008}. 

The dash-dotted line in Fig.~\ref{U-MC} is the analytical result $U_{RS}$ obtained using 
the replica-symmetric (RS) approximation (see section~\ref{rs-section}). 
Using Eq.~\eqref{rs-F} and Eq~\eqref{U-def}, $U_{RS}$  is obtained as   
%
\begin{equation}
U_{RS}=-N\beta(1+q_0^2-2q_0'^2). 
\label{U}
\end{equation}
% 
Here $q_0,q_0'$ are solutions of the saddle point equations Eqs.~\eqref{RS-saddle-1}
\eqref{RS-saddle-2}. Notice that $U_{RS}$ depends on $\omega$ only through $q_0,q_0'$. 
From Fig.~\ref{U-MC} one has that, while $U_{RS}$ is in perfect agreement with the 
numerics for $\beta\approx\beta_c$, deviations appear in the low-temperature region. 
Notice that already at $\beta\gtrsim 1$, $U_{RS}$ is incompatible with the data. 
Moreover, these deviations increase upon lowering the temperature. This has to be 
attributed to the replica symmetry breaking happening in the glassy phase. 
Finally, since in the limit $\beta\to\infty$ all the physical replicas are in the same state, 
one should expect that $u_\infty(\omega,\beta)\to n \tilde u_\infty$ (horizontal line in 
Fig.~\ref{U-MC}), with $\tilde u_{\infty}=-0.76321...$~\cite{parisi-1979,parisi-1983} the 
zero-temperature internal energy density of the S-K model on the plane. Surprisingly, 
this behavior is already observed in the MC data at $\beta\approx 3$, whereas it is not 
captured correctly by the RS approximation.


%%%%%%%%%%%%%%%%%%%%%%%%%%%%%%%%%
\begin{figure}[t]
\includegraphics*[width=0.93\linewidth]{./draft_figs/U_extrapolated}
\caption{
 The S-K model on the 2-sheets booklet and zero magnetic field: 
 The internal energy per spin $U/N$ in the thermodynamic limit. 
 The symbols are the Monte Carlo data extrapolated to the thermodynamic 
 limit $N\to\infty$, at fixed booklet ratio $\omega$ (see Fig.~\ref{cartoon}).  
 $U/N$ is plotted as a function of inverse temperature $\beta$ and for 
 $\omega=0,1/4,1/2,3/4,1$. The lines are the analytical results $U_{RS}$ 
 obtained using the replica-symmetric (RS) approximation. The stars denote 
 the value of $U_c/N$ at the paramagnetic-glassy transition. 
}
\label{U-RS}
\end{figure}
%%%%%%%%%%%%%%%%%%%%%%%%%%%%%%%%%%

%%%%%%%%%%%%%%%%%%%%%%%%%%%%%%%%%
\begin{figure}[t]
\includegraphics*[width=0.93\linewidth]{./draft_figs/U_extrapolated_v1}
\caption{
 The S-K model on the 2-sheets booklet with zero magnetic field: 
 The internal energy per spin $U/N$ in the thermodynamic limit plotted 
 as a function of inverse temperature $\beta$. The plus symbols are the same 
 Monte Carlo data as in Fig.~\ref{U-RS}. The lines denote $U/N$ 
 in the replica-symmetric (RS) approximation (same as in Fig.~\ref{U-RS}). 
 The circles are the results  in the one-step replica symmetry breaking 
 (1-RSB) approximation.
}
\label{U-RSB-1}
\end{figure}
%%%%%%%%%%%%%%%%%%%%%%%%%%%%%%%%%%


%%%%%%%%%%%%%%%%%%%%%%%%%%%%%%%%%
\begin{figure*}[t]
\includegraphics*[width=0.93\linewidth]{./draft_figs/Renyi_MC_v2}
\caption{ The classical disorder-averaged R\'enyi entropy per spin $[S_2(\omega)]/N$  
 in the S-K model on the $2$-sheets booklet with zero magnetic field: 
 $[S_2(\omega)]/N$ as a function of the inverse temperature $\beta$. The different panels 
 correspond to different booklet aspect ratios $\omega=3/4,1/2,1/4$ (see Fig.~\ref{cartoon}). 
 The triangles, squares and rhombi denote the Monte Carlo results for 
 booklets with $N=32,64,128$ spins per sheet. The plus symbols are the numerical 
 extrapolations to the thermodynamic limit. The inset in panel (a) shows 
 $[S_2]/N$ at fixed $\beta=2$, plotted versus $N^{-2/3}$.The dotted line is 
 a linear fit. In all the panels  the dash-dotted lines are the  
 analytical results in the replica-symmetry (RS) approximation. 
 The circles are the analytical results obtained 
 using the one-step replica symmetry breaking (1-RSB) ansatz.
}
\label{Renyi-MC}
\end{figure*}
%%%%%%%%%%%%%%%%%%%%%%%%%%%%%%%%%%

The behavior of $U/N$ for $\omega\ne1/2$ is investigated in Fig.~\ref{U-RS},
plotting $U/N$ as a function of $\beta$ and for $\omega=0,1/4,1/2,3/4,1$. The plus symbols 
are the MC data extrapolated to the thermodynamic, at fixed $\omega$. Similar to Fig.~\ref{U-RS}, 
the extrapolations are done assuming $U/N(\omega,n,\beta)=u_{\infty}(\omega,n,\beta)+
c/N^{\phi}$, with $\phi=2/3$ irrespective of $\omega$. The stars in Fig.~\ref{U-RS} are 
the critical values $U_c/N$ at the paramagnetic-glassy transition. The lines in Fig.~\ref{U-RS} 
denote $U_{RS}$ (cf. Eq.~\eqref{U}). 
Notice that at high temperature, and for generic $n$ and $\omega$, Eq.~\eqref{logZ-ht} and Eq.~\eqref{U-def} 
give  
%
\begin{equation}
U_{RS}= -N\frac{\beta}{2}(\omega^2(n^2-n)+n)+{\mathcal O}(\beta^2), 
\label{ht-U}
\end{equation}
%
i.e., a linear behavior of $U/N$ as a function of $\beta$. For $\omega=0$ and $\omega=1$ 
this behavior is exact up to the critical point at $\beta=\beta_c$, meaning that the higher 
orders ${\mathcal O}(\beta^2)$ in Eq.~\eqref{ht-U} are zero. This is only an approximation 
at intermediate $0<\omega<1$. Both the behaviors in Eq.~\eqref{U} and Eq.~\eqref{ht-U} are 
clearly confirmed in Fig.~\ref{U-RS}. 

However, the RS result is not correct for $\beta>\beta_c$, where the replica symmetry breaking 
has to be taken into account. This is  more carefully 
discussed in Fig.~\ref{U-RSB-1}, focusing on the low temperature phase at $1\le\beta\le 2$. 
The plus symbols and the lines are the same as in Fig.~\ref{U-RS}. The circles denote the 
internal energy per spin $U_{1\textrm{-}RSB}/N$ as obtained using the $1$-step replica 
symmetry breaking (1-RSB) approximation (see section~\ref{rsb-1-section}). Specifically, 
from Eq.~\eqref{RSB-1-logZ} and Eq.~\eqref{U-def}, for $n=2$ a straightforward calculation 
gives 
%
\begin{equation}
U_{1\textrm{-}RSB}=-N\beta(1+q_0^2+2q_1'^2(m_1-1)-2q_0'^2m_1), 
\label{U'}
\end{equation}
%
where $q_0,q_0',q_1',m_1\in\mathbb{R}$ are solutions of the saddle point 
equations \eqref{RSB-1-saddle-a}-\eqref{RSB-1-saddle-d}. Clearly, Eq.~\eqref{U'} 
implies $U_{1\textrm{-}RSB}\to U_{RS}$ for $q_1'\to q_0'$, as expected. 
Moreover, for $\beta\approx\beta_c$ one has $q_1'\approx q_0'\approx0$, implying   
that $U_{1\textrm{-}RSB}\approx U_{RS}$, i.e. the effects of the replica symmetry 
breaking are negligible near the critical point. Interestingly, at low temperatures, where the 
RS approximation fails (see Fig.~\ref{U-RS}), $U_{1\textrm{-}RSB}$ 
is in good agreement with the Monte Carlo data, at least up to $\beta\approx 2$.  
This allows us to conclude that, although the 1-RSB ansatz (cf. Eq.~\eqref{q-rsb1}) 
is not correct for $\beta\to\infty$, it captures some of the effects 
of the replica symmetry breaking.  


%##########################################################################
\section{The classical R\'enyi entropies}
\label{Renyi-section}

We now turn to discuss the behavior of the disorder-averaged classical R\'enyi entropies 
(cf. Eq.~\eqref{renyi}). Here we restrict ourselves to the second R\'enyi 
entropy $[S_2]$. Due to the mean-field nature of the S-K model (cf. Eq.~\eqref{SK-ham}), 
there is no well defined boundary between the two parts $A$ and $B$ of the system 
(unlike in local spin models). This suggests the volume-law behavior $[S_2]\propto N$ 
for all temperatures. Thus, it is natural to consider the entropy per spin 
$[S_2]/N$. 

%%%%%%%%%%%%%%%%%%%%%%%%%%%%%%%%%e
\begin{figure*}[t]
\includegraphics*[width=0.93\linewidth]{./draft_figs/I2_MC_v1}
\caption{The classical disorder-averaged mutual information per spin $[{\mathcal I}_2]/N$ 
 in the S-K model on the $2$-sheets booklet with zero magnetic field: 
 $[{\mathcal I}_2]/N$ versus the inverse temperature $\beta$. The 
 different panels correspond to different booklet ratios $\omega=1/2,1/4,1/8$ 
 (see Fig.~\ref{cartoon}). The same scale is used on both axes in all 
 panels. The symbols are the Monte Carlo data 
 for systems with $N=32,64,128$ spins per sheet. The dash-dotted line is 
 the analytical result in the replica-symmetric (RS) approximation.  
}
\label{I2-MC}
\end{figure*}
%%%%%%%%%%%%%%%%%%%%%%%%%%%%%%%%%%

The Monte Carlo data for $[S_2]/N$  are shown in Fig.~\ref{Renyi-MC} plotted versus the 
inverse temperature $0.25\le\beta\le 2.5$. Some details on the Monte Carlo method used 
to calculate $[S_2]$ are provided in Appendix~\ref{mc-method}. The different panels 
correspond to the booklet ratios $\omega=3/4,1/2,1/4$ (see Fig.~\ref{cartoon}). In all panels the 
triangles, squares, and rhombi correspond to booklets with $N=32,64,128$ spins per sheet. 
Clearly, finite size effects are present, which increase upon lowering 
the temperature, as expected. In order to obtain $[S_2]/N$ in the thermodynamic limit 
we fit the data to the ansatz $[S_2]/N=s_2(\omega)+c'(\omega)/N^{\phi'}$, where 
$s_2(\omega)$ is the entropy per spin in the thermodynamic limit, $c'$ a constant, and 
$\phi'$ the exponent of the finite-size corrections. The plus symbols in Fig.~\ref{Renyi-MC} 
are the results of the fits. We should mention that the fits give $\phi'\approx 2/3$, 
which is the exponent of the scaling corrections of the free energy in the standard 
S-K model. This is not surprising, since $[S_2]$ is obtained as the difference 
$[S_2]\equiv [F(0,2,\beta)-F(\omega,2,\beta)]$ (cf. Eq.~\eqref{renyi}). 
Clearly, from Fig.~\ref{Renyi-MC} one has that in the thermodynamic limit $[S_2]/N$ is 
finite for any $\beta$, confirming the expected volume law behavior. Moreover, $[S_2]/N$ 
exhibits a maximum in the infinite-temperature limit $\beta\to0$. The height of this 
maximum  is a decreasing function of $\omega$ (compare the panels (a)(b)(c) in 
Fig.~\ref{Renyi-MC}). 

The dash-dotted line denotes the analytical result $[S_2^{RS}]$ obtained within the RS 
approximation (see section~\ref{rs-section}). More precisely, $[S_2^{RS}]/N$ is obtained 
from Eq.~\eqref{renyi} and the expression for the free energy $[F_{RS}]$ (cf. Eq.~\eqref{rs-F}). 
Notice that in the high-temperature limit $\beta\to0$, where the RS approximation is exact, 
Eq.~\eqref{renyi} and Eq.~\eqref{ht-logZ} give 
%
\begin{equation}
[S^{RS}_2]=\omega\log(2)-\frac{\beta^2}{4}\omega^2+{\mathcal O}(\beta^3).
\end{equation}
%
From Fig.~\ref{Renyi-MC} one has that the extrapolated MC data are in quantitative 
agreement with $[S_2^{RS}]/N$ for $\beta\lesssim 1.5$, whereas strong deviations are 
observed at lower temperatures (not shown in the Figure). A better approximation for 
$[S_2]/N$ at low temperatures is obtained by including the effects of the replica 
symmetry breaking. The circles in Fig.~\ref{Renyi-MC} denote the one-step replica 
symmetry breaking result $[S_2^{1\textrm{-}RSB}]/N$ (see section~\ref{rsb-1-section}), 
which is obtained from Eq.~\eqref{renyi}  and Eq.~\eqref{RSB-1-logZ}. Remarkably, 
$[S_2^{1\textrm{-}RSB}]/N$ is in excellent agreement with the extrapolated Monte Carlo 
data for $\beta\lesssim 2$. 


%##########################################################################
\section{The classical R\'enyi mutual information}
\label{I2-section}

Here we focus on the behavior of the R\'enyi mutual information $[{\mathcal I}_2]$. 
Similar to $[S_2]$, the mutual information exhibits the volume law $[{\mathcal I}_2]
\propto N$. This is in contrast with local models, where $[{\mathcal I}_n]$ for any 
$n$ by construction obeys an area law at all temperatures. Here we consider the 
mutual information per spin $[{\mathcal I}_2]/N$. 

Figure~\ref{I2-MC} plots $[{\mathcal I}_2]/N$ versus $0\le\beta\le 2.5$ and $\omega=1/2,1/4,
1/8$ (panels from left to right in the Figure). Notice that by definition (cf. Eq.~\eqref{MI}) 
$[{\mathcal I}_n(\omega)]=[{\mathcal I}_n(1-\omega)]$. Circles, squares, and triangles  
are Monte Carlo data for $N=32,64,128$. In the high-temperature region $[{\mathcal I}_2]$ 
exhibits a vanishing behavior. Moreover, finite size effects are ``small''. Using Eq.~\eqref{logZ-ht} 
it is straightforward to derive the high-temperature behavior of $[{\mathcal I}_n]$ as  
%
\begin{equation}
[{\mathcal I}_n]=\frac{N\beta^2}{2}\omega(1-\omega)n+{\mathcal O}
(\beta^4).
\end{equation}
%
Interestingly, $[{\mathcal I}_2]/N$ increases upon lowering the temperature 
up to $\beta\approx 1$, where it exhibits a maximum. The height of this maximum 
decreases as a function of $\omega$. We numerically observe that its position 
is not simply related to the paramagnetic-glassy transition. Furthermore, the 
data for $[{\mathcal I}_2]/N$ at different system sizes do not exhibit any crossing. This is 
in sharp contrast with local models~\cite{jaconis-2013}, where ${\mathcal I}_n/L$ exhibits a 
crossing at a second order phase transition. The dash-dotted line in Fig.~\ref{I2-MC} is the 
analytical result obtained using the replica symmetric (RS) approximation (see 
section~\ref{rs-section}). Formally, this is obtained using Eq~\eqref{MI} and Eq.~\eqref{rs-F}, 
and it is in perfect agreement with the MC data in the whole paramagnetic phase.  


Interestingly, at low temperatures $[{\mathcal I}_2]/N$ exhibits strong finite-size corrections, 
and significant deviations from the RS result. In order to extract the thermodynamic behavior of 
$[{\mathcal I}_2]/N$ we fit the MC data to 
%
\begin{equation}
\frac{[{\mathcal I}_2]}{N}=a+\frac{b}{N^{\phi''}},
\end{equation}
%
where we fix $\phi''=2/3$. The results of the fits are shown in Fig.~\ref{I2-extrapolated}.  
Different symbols now correspond to different aspect ratios $\omega$. Remarkably, the RS 
approximation (dash-dotted lines) is 
in good agreement with the extrapolations up to $\beta\lesssim 1.5$, whereas deviations 
occur at lower temperatures (not shown in the figure). 
For $\omega=1/8$, the numerical results exhibit deviations from the RS result already at $\beta\gtrsim 1$. 
These deviations have to be attributed to the physics of the replica symmetry breaking. 
The full rhombi in Fig.~\ref{I2-extrapolated} denote the one-step replica symmetry breaking result 
$[{\mathcal I}_2^{1\textrm{-}RSB}]$, which is obtained using Eq.~\eqref{MI} and Eq.~\eqref{RSB-1-logZ}). 
The agreement between $[{\mathcal I}_2^{1\textrm{-}RSB}]$ and the Monte Carlo data is perfect 
up to $\beta\lesssim 2$. 
 


%##########################################################################
\section{Summary and Conclusions}
\label{conclusions}

We investigated the \emph{classical} R\'enyi entropy $S_n$ and the mutual information 
${\mathcal I}_n$ in the Sherrington-Kirkpatrick (S-K) model, which is the paradigm model 
of mean-field spin glasses. We focused on the quenched 
averages $[S_n]$ and $[{\mathcal I}_n]$, with the brackets $[\cdot]$ denoting the average 
over different disorder realizations. Specifically, here $[S_n]$ and $[{\mathcal I}_n]$ 
were obtained from suitable combinations of the partition functions of the S-K model on the 
$n$-sheets booklet (cf. Fig.~\ref{cartoon}). This is constructed by ``gluing'' together 
$n$ independent replicas (``sheets'') of the model. On each replica the spins are 
divided into two groups $A$ and $B$, containing $N_A$ and $N_B$ spins respectively. The 
spins in part $A$ of the different sheets are identified. Due to the mean-field nature of 
the model, physical quantities depend on the bipartition only through the aspect ratio 
$\omega\equiv N_A/N$. 

%%%%%%%%%%%%%%%%%%%%%%%%%%%%%%%%%
\begin{figure}[t]
\includegraphics*[width=0.93\linewidth]{./draft_figs/I2_extrapolated}
\caption{ The classical mutual information per spin $[{\mathcal I}_2]/N$ 
 in the SK model on the $2$-sheets booklet with zero magnetic field: 
 $[{\mathcal I}_2]/N$ plotted versus $\beta$. The symbols denote the 
 Monte Carlo results extrapolated to the thermodynamic limit for 
 booklet ratios $\omega=1/2,1/4,1/8$ (circles, squares, triangles). 
 The dash-dotted lines are the analytical results in the 
 replica-symmetric (RS) approximation. The full symbols (rhombi) 
 are the results in the first-step replica-symmetry-breaking (1-RSB) 
 approximation.
}
\label{I2-extrapolated}
\end{figure}
%%%%%%%%%%%%%%%%%%%%%%%%%%%%%%%%%%


We first investigated the phase diagram of the S-K model on the $2$-sheets booklet as a 
function of $\omega$ and the inverse temperature $\beta$. We considered the situation 
without external magnetic field ($h=0$ in Eq.~\eqref{SK-intro}). For any $\omega$, in the 
thermodynamic limit, the model exhibits a glassy phase for low enough temperatures and 
a standard paramagnetic one at high temperature. The two phases are divided 
by a phase transition, as in the S-K model on the plane. Interestingly, the critical 
temperature exhibits a non-trivial dependence on $\omega$ that we determined analytically. 
We fully characterized the high-temperature phase of the model using the replica-symmetric 
(RS) approximation. Specifically, we provided the exact analytical expression for the 
disorder-averaged free energy $[F_{para}(\omega,n,\beta)]$  (cf. Eq.~\eqref{ht-logZ}) in 
the thermodynamic limit. On the other hand, we observed that in the glassy phase the 
replica symmetry has to be broken, similar to the standard S-K model. To take into account 
the effects of the replica symmetry breaking we used a generalization of the Parisi 
ansatz~\cite{parisi-1980}, i.e., breaking the replica symmetry in successive 
steps. We restricted ourselves to the one-step replica symmetry breaking ($1$-RSB),   
providing the exact expression for the disorder-averaged free energy $F_{1\textrm{-}RSB}
(\omega,n,\beta)$ (cf. Eq.~\eqref{RSB-1-logZ}). 

Finally, we compared our analytical results with Monte Carlo simulations, focusing on the 
internal energy per spin $U(\omega,n,\beta)/N$. In the high-temperature region, where the 
replica symmetry is not broken, we observed perfect agreement with the replica-symmetric 
result $U_{RS}/N$, for any $\omega$. In the glassy phase at low temperature, we found 
significant deviations from $U_{RS}$ already at $\beta\approx 1$. On the other hand,  
the one-step replica-symmetry-broken approximation $U_{1\textrm{-}RSB}/N$ matched the 
Monte Carlo data for $\beta\lesssim 3$. 

Finally, we discussed the behavior of $[S_2]$ and $[{\mathcal I}_2]$. Due to the mean-field 
nature of the S-K model we observed the volume laws $[S_2]\propto N$ and 
$[{\mathcal I}_2]\propto N$, for any $\beta$. We considered the densities 
$[S_2]/N$ and $[{\mathcal I}_2]/N$. Using the replica-symmetric (RS) and the one-step 
replica symmetry breaking ($1$-RSB) approximations we provided analytical results for 
both $[S_2]/N$ and $[{\mathcal I}_2]/N$. We observed that at high temperature the 
RS result fully agrees with the Monte Carlo simulations, whereas deviations occur in the 
glassy phase, signalling the replica symmetry breaking. Similar to the internal energy, 
we found that the $1$-RSB approximation fully matches the Monte Carlo data for $\beta\approx 3$.
Interestingly, we numerically observed that at the critical point ${\mathcal I}_2/N$ 
does not show any crossing for different system sizes, in contrast with local 
models~\cite{jaconis-2013}.

Our work opens several research directions. First, it would be interesting to extend our 
results taking into account the full breaking of the replica symmetry, i.e., going beyond 
the one-step replica symmetry breaking approximation. This would allow us to reach a conclusion 
on the correctness of the replica symmetry breaking scheme that we used. Moreover, it 
would be interesting to discuss the finite size-corrections to the saddle 
point approximation. An interesting direction would be to investigate whether the 
glassy behavior is reflected in the volume-law corrections to the classical R\'enyi 
entropies. Finally, it would be interesting to extend our results to {\it quantum} spin 
systems exhibiting glassy behavior and replica symmetry breaking~\cite{read-1995,
andreanov-2012}. 

\section{Acknowledgements}
We would like to thank Pasquale Calabrese for useful discussions. V.A.  acknowledge 
financial support by the ERC under Starting Grant 279391 EDEQS. 
[Stephen's grant under Lode?]




\appendix

%##########################################################################
\section{The saddle point equations} 
\label{saddle-equations}

In this section we provide the analytical expression for the saddle point equations 
\eqref{saddle-final}\eqref{saddle-final-a}, which determine the overlap tensor 
$q_{\gamma\gamma'}^{rr'}$ (see section~\ref{replica-sec}). We restrict ourselves 
to zero magnetic field and to the $2$-sheets booklet (see Fig.~\ref{cartoon}). It is 
straightforward to generalize the calculation to the case with non zero magnetic 
field and to the $n$-sheets booklet. Specifically, here we provide the saddle point 
equations obtained in both the replica-symmetric (RS) (see section~\ref{rs-section}), 
and the one-step replica symmetry breaking ($1$-RSB) approximation (see 
section~\ref{rsb-1-section}). 

%####################################
\subsection{The replica-symmetric (RS) approximation}

In the replica-symmetric approximation $q_{\gamma\gamma'}^{rr'}$ depends on the two 
parameters $q_0,q_0'\in\mathbb{R}$ (cf. Eq.~\eqref{rs-ansatz}). The saddle point 
equations are easily derived from the RS approximation for the free energy $[F_{RS}
(\omega,n,\beta)]$ (cf. Eq.~\eqref{rs-F}) as $\nabla_{\mathbf{q}}[F_{RS}(\omega,2,
\beta)]=0$, where $\mathbf{q}\equiv(q_0,q_0')$, and $\nabla_{\mathbf{q}}\equiv
\partial/\partial\mathbf{q}$. A straightforward calculation gives the two equations 
%
\begin{equation}
\label{RS-saddle-1}
\frac{1-q_0}{1-\omega}=2\int dz G_0(z)(1+\exp(2\beta^2(q_0-q_0'))\cosh(2z))^{-1},
\end{equation}
%
and
%
\begin{multline}
\label{RS-saddle-2}
2\beta^2\frac{1-q_0'}{1-\omega}=
\int dz G_0(z)\Big\{\frac{\omega}{1-\omega}\Delta_0(z)
\log\cosh(2z)\\
+\Delta_0(z)\log(1+\exp(2\beta^2(q_0-q_0'))\cosh(2z))\\
+2\beta^2(1+\exp(2\beta^2(q_0-q_0'))\cosh(2z))^{-1}
\Big\},
\end{multline}
%
where we defined $\Delta_0(z)$ as 
%
\begin{equation}
\Delta_0(z)\equiv\frac{z^2}{2\beta^2q_0'^2}-\frac{1}{2q_0'},
\end{equation}
%
and the so-called heat kernel $G_0(z)$ as 
%
\begin{equation}
G_0(z)\equiv\frac{1}{\sqrt{2\pi \beta^2 q_0'}}\exp\Big(-\frac{z^2}
{2\beta^2 q_0'}\Big).
\end{equation}
%
Notice that $\int dz G_0(z)\Delta_0(z)=0$.

%##########################################################################
\subsection{The one-step replica symmetry breaking ($1$-RSB) approximation}

In the one-step replica symmetry breaking approximation (see section~\ref{rsb-1-section}) 
$q_{\gamma\gamma'}^{rr'}$ depends on the four parameters $q_0,q_0',q_1',m_1\in\mathbb{R}$. 
The saddle point equations are given as $\nabla_{\mathbf{p}}[F_{1\textrm{-}RSB}]=0$, 
where $\mathbf{p}\equiv(q_0,q_0',q_1',m_1)$, and $[F_{1\textrm{-}RSB}]$ is 
the disorder-averaged free energy given in Eq.~\eqref{RSB-1-logZ}.
It is useful to define the modified heat kernel $G_1(z)$ as 
%
\begin{equation}
G_1(z)\equiv\frac{1}{\sqrt{2\pi \beta^2 (q_1'-q_0')}}\exp\Big(-\frac{z^2}
{2\beta^2 (q_1'-q_0')}\Big), 
\end{equation}
%
and 
%
\begin{align}
& \Delta_1(z)\equiv \frac{z^2}{\beta^2(q_1'-q_0')^2}-\frac{1}{q_1'-q_0'}\\\nonumber
& \Gamma(z)\equiv \Big\{\int dz'G_1(z')\cosh^{m_1}(2z+2z')\Big\}^{-1}\\\nonumber
& \Gamma'(z)\equiv \Big\{\int dz'G_1(z')\Big(1+\cosh(2z+2z')\Big)^{m_1}\Big\}^{-1}\\\nonumber
& \Theta(z,z')\equiv 1+\exp(2\beta^2(q_0-q_1'))\cosh(2z+2z'). 
\end{align}
%
Finally, the saddle point equations for $q_0,q_0',q_1',m_1$ are obtained as 
%
\begin{widetext}
\begin{multline}
0=(1+q_0-2\omega)\exp(-2\beta^2(q_0-q_1'))
-2(1-\omega)\int dzdz'G_{0}(z)
\Gamma'(z)G_{1}
(z')\cosh(2z+2z')\Theta^{m_1-1}(z,z')
\label{RSB-1-saddle-a}
\end{multline}
%
\begin{multline}
0=4\beta^2m_1q_0'+\frac{\omega}{m_1}\int dz G_{0}(z)
\Delta_0(z)\log\int dz'G_{1}(z')
\cosh^{m_1}(2z +2z')\\
+\frac{1-\omega}{m_1}\int dzG_{0}(z)
\Delta(z)\log\int dz'G_{1}(z')\Theta^{m_1}(z,z')
-\frac{\omega}{m_1}\int dzdz'G_{0}(z)\Gamma(z)
G_{1}(z')\Delta_1(z')\cosh^{m_1}(2z+2z')\\
-\frac{1-\omega}{m_1}
\int dzdz'G_{0}(z)\Gamma'(z)G_{1}(z')\Delta_1(z')\Theta^{m_1}(z,z')
\label{RSB-1-saddle-b}
\end{multline}
%
\begin{multline}
0=-4\beta^2(m_1-1)q_1'-4\beta^2\omega+\frac{\omega}{m_1}
\int dzdz'G_{0}(z)\Gamma(z)G_{1}(z')\Delta_1(z')
\cosh^{m_1}(2z+2z')\\+
\frac{1-\omega}{m_1}\int dzdz'G_{0}(z)
\Gamma'(z)
G_{1}(z')\Big\{\Delta_1(z')\Theta^{m_1}(z,z')
-4\beta^2m_1\exp(2\beta^2(q_0-q_1'))\cosh(2z+2z')
\Theta^{m_1-1}(z,z')
\Big\}
\label{RSB-1-saddle-c}
\end{multline}
%
\begin{multline}
0=-\beta^2(q_1'^2-q_0'^2)+\frac{\omega}{m_1}
\int dzdz'G_{0}(z)\Gamma(z)G_{1}(z')\cosh^{m_1}(2z+2z')
\log\cosh(2z+2z')\\+
\frac{1-\omega}{m_1}\int dzdz' G_{0}(z)\Gamma'(z)
G_{1}(z')\Theta^{m_1}(z,z')\log(\Theta(z,z'))
-\frac{1-\omega}{m_1^2}\int dz G_{0}(z)\log\int dz' 
G_{1}(z')\Theta^{m_1}(z,z')\\
-\frac{\omega}{m_1^2}\int dzG_{0}(z)\log
\int dz'G_{1}(z')\cosh^{m_1}(2z+2z').
\label{RSB-1-saddle-d}
\end{multline}
\end{widetext}
%

%##########################################################################
\section{Monte Carlo method to calculate the classical R\'enyi entropies}
\label{mc-method}

Here we describe the Monte Carlo method that we used to calculate the R\'enyi entropies 
$S_n$ and the mutual information ${\mathcal I}_n$. The method exploits the representation 
of $S_n$ as ratio of partition functions as 
%
\begin{equation}
\label{renyi-app}
S_n(A)\equiv \frac{1}{1-n}\log\left(\frac{Z(A,n,\beta)}{Z^n(\beta)}\right),
\end{equation}
%
where $Z(A,n,\beta)$ and $Z(\beta)$ are the partition functions of the model on the booklet and 
on the plane, respectively (see Fig.~\ref{cartoon} and section~\ref{booklet} for the definitions).  
There are only few approaches to numerically calculate the ratio of partition functions in 
Eq.~\eqref{renyi-app}. For instance, a brute force numerical integration of the internal energy as 
a function of temperature\cite{jaconis-2013} can be used to calculate $Z(A,n,\beta)$ and $Z(\beta)$. 
However, this requires high accuracy over a large range of temperature.

Here we directly calculate the ratio of partition functions in Eq.~\eqref{renyi-app} using the 
so-called \emph{ratio trick}. In the ratio trick one splits the subsystem $A$ as $A=\sum_{i}A_i$, 
where $A_i$ is a sequence of subsystems such that $A_i\subset A_{i+1}$. 
Then one writes 
%
\begin{align}
\label{ratio-trick}
\frac{Z(A,n,\beta)}{ Z^n(\beta)} = \prod_{i=0}^{N/4-1} \frac{Z(A_i,n,\beta)}{Z(A_{i+1},n,\beta)},
\end{align}
%
where for simplicity we specialized to intervals $A_i$ such that
%
\begin{align}
\label{sub-choice}
A_i = 4i,\quad i \in [0,N/4] .
\end{align}
%
Crucially, if the length of $A_i$ increases mildly with $i$, each term in the product 
in the left hand side of Eq.~\eqref{ratio-trick} can be sampled efficiently in Monte 
Carlo~\cite{stephan-2014}. Notice that the same trick has also been used in 
Ref.~\onlinecite{alba-2010,alba-2011,alba-2013}. More specifically, one can write
%
\begin{align}
\frac{Z(A_i,n,\beta)}{Z(A_{j},n,\beta)} = \frac{T_{j\rightarrow i}}{T_{i\rightarrow j}}
\end{align}
%
where $T_{i\rightarrow j}$  is the (Monte Carlo) transition probability from a spin 
configuration living on the booklet with $A=A_i$ to a spin configuration living on the 
booklet with $A=A_j$. Notice that spins in region $A$ of different sheets are identified 
(cf. Eq.~\eqref{book-constraint}). Clearly, if $j<i$ one has $T_{i\rightarrow j}=1$.
When $j>i$ a naive method to determine $T_{i\rightarrow j}$ would be to simply count the 
fraction of times (during the Monte Carlo update) that a spin configuration living on 
the booklet with $A=A_i$ is also a valid spin configuration on the booklet with $A=A_j$.
In the following we provide a more efficient scheme to calculate $T_{i\rightarrow j}$.

To summarize, our approach for calculating the ratio of partition functions in 
Eq.~\eqref{ratio-trick} consists of four steps:
\begin{enumerate}
\item Do a full Monte Carlo sweep of the system on the booklet with $A=A_i$ 
to generate an importance sampled spin configuration. This can be done with any 
update scheme, such as standard Metropolis or cluster updates
\item For the spins living in $A_{i+1}-A_i$ calculate 
%
\begin{align}
W_a = \sum_i e^{-\beta E_i}.
\end{align}
%
Here (cf. Eq.~\eqref{sub-choice}) $i$ runs over all the $n2^4$ configurations of the 
four spins in $A_{i+1}-A_i$, and $E_i$ is the energy associated with configuration $i$. 
\item Calculate the quantity
\begin{align}
W_b = \sum_j e^{-\beta E_j}
\end{align}
where $j$ runs over all the $2^4$ configurations with the four spins living in $A_{i+1}-
A_i$ and different sheets are identified.
\item Calculate the transition probability $T_{i\rightarrow i+1}$ as
%
\begin{align}
T_{i \rightarrow i+1} = \Bigl< \frac{W_b}{W_a} \Bigr>.
\end{align}
\end{enumerate}
%
This method is much faster than simply counting configurations as it allows 
us to integrate over all possible configurations of the spins in $A_{i+1}-A_i$, 
treating the remaining ones as a bath that is updated by the regular Monte Carlo update. 
The computational cost of this procedure grows exponentially with the number of spins 
in $A_{i+1}-A_i$. 
In our Monte Carlo simulations we calculate $Z(A_i,n,\beta)/Z(A_{i+1},n,\beta)$ 
for every disorder realization and $S_n(A)$ using Eq.~\eqref{renyi-app}. 
Finally, we average over the disorder to obtain $[S_n(A)]$.


%##########################################################################
\section{The critical temperature}
\label{binder}

In this section we numerically determine the critical temperature $T_c$ of the 
paramagnetic-glassy transition in the S-K model on the booklet. We restrict 
ourselveses the $2$-sheets booklet with fixed aspect ratio (cf. Eq.~\eqref{a-ratio}) 
$\omega=1/4$. 

Given two identical, i.e., with the same disorder realization, and independent 
copies of the booklet, the overlap order parameter $q_i$ is defined as 
%
\begin{equation}
q\equiv\frac{1}{2N}\sum\limits_{i}\langle S_i^{(r,1)}S_i^{(r,2)}\rangle,
\end{equation}
%
where $S_i^{(r,1)}$ and $S_i^{(r,2)}$ are the spins living on the two copies 
of the booklet and $r=1,2$ labels the booklet sheets in each copy. The 
angular brackets $\langle\cdot\rangle$ denote the thermal average at 
fixed disorder realization. 
The overlap Binder cumulant $U_4$, which is defined as 
%
\begin{equation}
U_4\equiv \frac{1}{2}\left(3-\frac{[\langle q^4\rangle]}{[\langle q^2\rangle]^2}
\right)
\end{equation}
% 


%%%%%%%%%%%%%%%%%%%%%%%%%%%%%%%%%
\begin{figure}[t]
\includegraphics*[width=0.87\linewidth]{./draft_figs/Binder}
\caption{ 
}
\label{binder-fig}
\end{figure}
%%%%%%%%%%%%%%%%%%%%%%%%%%%%%%%%%%


\begin{thebibliography}{99}
\bibitem{binder-1986}
K.~Binder and A.~P.~Young, Rev.\ Mod.\ Phys.\ {\bf 58}, 801 (1986).

\bibitem{parisi-book}
M.~Mezard, G.~Parisi, and M.~Virasoro, \emph{Spin Glass theory and beyond}, World Scientific, Singapore (1987).

\bibitem{young-1998}
A.~P.~Young, \emph{Spin Glasses and Random Fields} (Singapore: World Scientific).


\bibitem{nishimori-book}
H.~Nishimori, \emph{Statistical Physics of Spin Glasses and Information Processing}, 
Clarendon Press, Oxford (2001).

\bibitem{castellani-2005}
T.~Castellani and A.~Cavagna, J.\ Stat.\ Mech.\ (2005) P05012. 

\bibitem{sherrington-1978}
D.~Sherrington and S.~Kirkpatrick, Phys.\ Rev.\ B {\bf 17}, 4385 
(1978).

\bibitem{sherrington-1978-prl}
D.~Sherrington and S.~Kirkpatrick, Phys.\ Rev.\ Lett.\ {\bf 35}, 
1972 (1978).

\bibitem{parisi-1980}
G.~Parisi, J.\ Phys.\ A\ {\bf 13}, 1101 (1980). 

\bibitem{talagrand-2006}
M.~Talagrand, Ann.\ of\ Math.\ {\bf 163}, 221 (2006). 

\bibitem{pastur-1991}
L.~A.~Pastur and M.~V.~Shcherbina, J.\ Stat.\ Phys.\ {\bf 62}, 1 (1991).

\bibitem{mezard-1984}
M.~Mezard, G.~Parisi, N.~Sourlas, G.~Toulouse, and M.~Virasoro, 
Phys.\ Rev.\ Lett.\ {\bf 52}, 1156 (1984). 

\bibitem{rammal-1986}
R.~Rammal, G.~Toulouse, and M.~A.~Virasoro, Rev.\ Mod.\ Phys.\ {\bf 58}, 765 
(1986).

\bibitem{cardy-book}
J.~Cardy, \emph{Scaling and Renormalization in Statistical Physics} 
(Cambridge Lecture Notes in Physics). 

\bibitem{yucesoy-2012}
B.~Yucesoy, H.~G.~Katzgraber, and J.~Machta, Phys.\ Rev.\ Lett. {\bf 109}, 
177204 (2012).

\bibitem{billoire-2012}
A.~Billoire, L.~A.~Fernandez, A.~Maiorano, E.~Marinari, 
V.~Martin-Mayor, G.~Parisi, F.~Ricci-Tersenghi, J.~J.~Ruiz-Lorenzo, 
and D.~Yllanes, Phys.\ Rev.\ Lett.\ {\bf 110}, 219701 (2013).

\bibitem{yucesoy-2013}
B.~Yucesoy, H.~G.~Katzgraber, and J.~Machta, Phys.\ Rev.\ Lett.\ {\bf 110}, 
219702 (2013)

\bibitem{morrison-2008}
S.~Morrison, A.~Kantian, A.~J.~Daley, H.~G.~Katzgraber, M.~Lewenstein, 
H.~P.~B\"uchler, and P.~Zoller, New\ J.\ Phys.\ {\bf 10}, 
073032 (2008). 

\bibitem{rotondo-2015}
P.~Rotondo, E.~Tesio, and S.~Caracciolo, Phys.\ Rev.\ B {\bf 91}, 
014415 (2015). 


\bibitem{ghofraniha-2015}
N.~Ghofraniha, I.~Viola, F.~Di~Maria, G.~Barbarella, G.~Gigli, L.~Leuzzi, 
and C.~Conti, Nature\ Comm.\ {\bf 6}, 6058 (2015). 

\bibitem{amico-2008}
L. Amico, R. Fazio, A. Osterloh, and V. Vedral, Rev.\ Mod.\ Phys.\ {\bf 80}, 517 (2008).

\bibitem{eisert-2009}
J. Eisert, M. Cramer, and M. B. Plenio, Rev.\ Mod.\ Phys.\ {\bf 82}, 277 (2009).


\bibitem{calabrese-2009}
P.~Calabrese, J.~Cardy, and B. Doyon Eds., Special issue: Entanglement 
entropy in extended systems, J.\ Phys.\ A {\bf 42}, 50 (2009).


\bibitem{cc-rev}
P.~Calabrese and J.~Cardy, J.\ Phys.\ A {\bf 42} 504005 (2009).

\bibitem{holzhey-1994} C. Holzhey, F. Larsen, and F. Wilczek,
Nucl. Phys. B {\bf 424}, 443 (1994).


\bibitem{vidal-2003}
G.~Vidal, J.~I.~Latorre, E.~Rico, and A.~Kitaev, Phys.\ Rev.\ Lett.\ 
{\bf 90}, 227902 (2003). J.~I.~Latorre, E.~Rico, and G.~Vidal, Quant.\ Inf.\ 
and\ Comp.\ {\bf 4}, 048 (2004).


\bibitem{calabrese-2004}
P.~Calabrese and J.~Cardy, J.\ Stat.\ Mech.\ (2004) P06002. 
P.~Calabrese and J.~Cardy, Int. J.\ Quant.\ Inf.\ {\bf 4}, 429 (2006).

\bibitem{calabrese-2012}
P.~Calabrese, J.~Cardy, and E.~Tonni, Phys.\ Rev.\ Lett.\ {\bf 109}, 
130502 (2012). 

\bibitem{jaconis-2013}
J.~Iaconis, S.~Inglis, A.~B.~Kallin, and R.~G.~Melko, 
Phys.\ Rev.\ B {\bf 87}, 195134 (2013).

\bibitem{stephan-2014}
J.-M.~St\'ephan, S.~Inglis, P.~Fendley, and R.~G.~Melko, 
Phys.\ Rev.\ Lett.\ {\bf 112}, 127204 (2014).

\bibitem{alcaraz-2013}
F.~C.~Alcaraz and M.~A.~Rajabpour, Phys.\ Rev.\ Lett.\ {\bf 111}, 017201 (2013).

\bibitem{stephan-2014-a}
J.-M.~St\'ephan, Phys.\ Rev.\ B {\bf 90}, 045424 (2014). 

\bibitem{wolf-2008}
M.~M.~Wolf, F.~Verstraete, M.~B.~Hastings, and J.~I.~Cirac, Phys.\ Rev.\ Lett.\ 
{\bf 100}, 070502 (2008). 

\bibitem{refael-2009}
G.~Refael and J.~E.~Moore J.\ Phys.\ A:\ Math.\ Theor.\ {\bf 42} 504010 (2009).

\bibitem{castelnovo-2010}
C.~Castelnovo, C.~Chamon, and D.~Sherrington, Phys.\ Rev.\ B {\bf 81}, 184303 (2010).

\bibitem{parisi-1979}
G.~Parisi, Phys.\ Rev.\ Lett.\ {\bf 43}, 1754 (1979).

\bibitem{guerra-2002}
F.~Guerra and F.~L.~Toninelli, Commun.\ Math.\ Phys.\ {\bf 230}, 71 (2002). 

\bibitem{almeida-1978}
J.~R.~L.~de Almeida and D.~J.~Thouless, J.\ Phys.\ A {\bf 11}, 983 (1978). 

\bibitem{billoire-2007}
A.~Billoire, in \emph{Rugged Free Energy Landscape}, Springer Lecture 
Notes in Physics, edited by W. Janke (Springer, Berlin-Heidelberg 
2007. 

\bibitem{aspelmeier-2008}
T.~Aspelmeier, A.~Billoire, E.~Marinari, and M.~A.~Moore, 
J.\ Phys.\ A:\ Math.\ Theor.\ {\bf 41} 324008 (2008). 

\bibitem{parisi-1983}
G.~Parisi, Phys.\ Rev.\ Lett\ {\bf 50}, 1946 (1983). 

\bibitem{read-1995}
N.~Read, S.~Sachdev, and J.~Ye, Phys.\ Rev.\ B {\bf 52}, 384 (1995). 

\bibitem{andreanov-2012}
A.~Andreanov and M.~M\"uller, Phys.\ Rev.\ Lett.\ {\bf 109}, 177201 (2012). 

\bibitem{alba-2010}
V.~Alba, L.~Tagliacozzo, and P.~Calabrese, Phys.\ Rev.\ B {\bf 81}, 060411(R) (2010).

\bibitem{alba-2011}
V.~Alba, L.~Tagliacozzo, and P.~Calabrese, J.\ Stat.\ Mech.\ (2011) P06012. 

\bibitem{alba-2013}
V.~Alba, J.\ Stat.\ Mech.\ (2013) P05013. 






%\bibitem{katzgraber-2003}

%H.~G.~Katzgraber and I.~A.~Campbell, 

%Phys.\ Rev.\ B {\bf 68}, 180402R (2003).

\end{thebibliography}

%Unused bibitems

%\bibitem{newman-1996}
%C.~Newman and D.~L.~Stein, Phys.\ Rev.\ Lett.\ {\bf 76}, 515 (1996).
%\bibitem{wilms-2012}
%J.~Wilms, J.~Vidal, F.~Verstraete,and S.~Dusuel, J.\ Stat.\ Mech.\ (2012)P01023. 
%
%\bibitem{parisi-1980-a}
%G.~Parisi, J.\ Phys.\ A\ {\bf 13}, 1887 (1980). 
%\bibitem{huse-1987}
%D.~S.~Fisher and D.~A.Huse, J.\ Phys.\ A:\ Math.\ Gen.\ {\bf 20} L997 (1987).
%\bibitem{wilms-2011}
%J.~Wilms, M.~Troyer, and F.~Verstraete, J.\ Stat.\ Mech.\ (2011) P10011.
%\bibitem{zaletel-2011}
%M.~P.~Zaletel, J.~H.~Bardarson, and J.~E.~Moore, Phys.\ Rev.\ Lett.\ {\bf 107}, 
%020402 (2011).
%\bibitem{huse-1987-a}
%D.~A.~Huse and D.~S.~Fisher, J.\ Phys.\ A:\ Math.\ Gen.\ {\bf 20} L1005 (1987).
%\bibitem{fisher-1988}
%D.~S.~Fisher and D.~A.Huse, Phys.\ Rev. B {\bf 38}, 386 (1988).
%\bibitem{parisi-1980-b}
%G.~Parisi, J.\ Phys.\ A\ {\bf 13}, L115 (1980). 
%\bibitem{fisher-1986}
%D.~S.~Fisher and D.~A.~Huse, Phys.\ Rev.\ Lett.\ {\bf 56} 1601 (1986).
%\bibitem{palassini-2000}
%M.~Palassini and A.~P.~Young, Phys.\ Rev.\ Lett.\  {\bf 85}, 3017 (2000).
%\bibitem{stephan-2010}
%J-M~St\'ephan, G.~Misguich, and V.~Pasquier, Phys.\ Rev.\ B, {\bf 82}, 125455 
%(2010); 
%\bibitem{stephan-2011}
%J-M~St\'ephan, G.~Misguich, and V.~Pasquier, Phys.\ Rev.\ B {\bf 84}, 
%195128 (2011).
%\bibitem{stephan-2009}
%J-M~St\'ephan, S.~Furukawa, G.~Misguich, and V.~Pasquier, Phys.\ Rev.\ B, {\bf 80}, 
%184421 (2009)
%\bibitem{krzakala-2000}
%F.~Krzakala and O.~C.~Martin, Phys.\ Rev.\ Lett.\ {\bf 85}, 3013 (2000).
\end{document}



